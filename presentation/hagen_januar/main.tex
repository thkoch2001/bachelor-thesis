\documentclass{beamer}
%\usepackage{ngerman}
\usepackage[utf8]{inputenc}
\usepackage[T1]{fontenc}
\usepackage{textcomp}
\usepackage{lmodern}
\usepackage{pgf}
\usepackage{verbatim}
\usepackage{listings}
\usepackage[]{hyperref} 
\usepackage{multicol}
\usepackage{wasysym}
\usepackage{tikz}

\setbeamersize{text margin left = 0.4em}
\setbeamersize{text margin right = 0.4em}
%\setbeamercovered{transparent}

\title{REST Schnittstelle für die Groupware Kolab}
\subtitle{Bachelorarbeit}
\date{\today}
\author{Thomas Koch}

\usetheme{boxes}
\usecolortheme{dolphin}

\AtBeginSection[]
{
   \begin{frame}
       \frametitle{Outline}
       \tableofcontents[currentsection]
   \end{frame}
}

\begin{document}
\begin{frame}
  \maketitle{}
\end{frame}

\begin{frame}{Hintergrund: Kolab}
  Freie Groupware, Auftragsentwicklung BSI, Auswärtiges Amt 2002

  Wiederverwendung bestehender Komponenten: Cyrus IMAP, OpenLDAP, Postfix, \ldots + viel Perl Magie

  Persistenz als Email-Anhang in IMAP Ordnern, Kommunikation über IMAP
\end{frame}

\begin{frame}{Schwerpunkte der Arbeit}
  \begin{itemize}
  \item Welche Komponenten der Implementierung können wiederverwendet werden, um verschiedene Medientypen zu unterstützen?
  \item Inwieweit unterstützen die Medientypen Hyperlinks, sind also durch den Client erkundbar ohne out-of-band knowledge?
  \end{itemize}  
\end{frame}

\begin{frame}{Persönliche Motivation}
  \begin{center}
    2012

    Mein Adressbuch \& Kalender sind aus Papier.

    IT Firmen mit freier Groupware: 0\%
  \end{center}
\end{frame}

\begin{frame}{Anforderungen an die Implementierung}
  \begin{itemize}
  \item Lesen/Schreiben der verwalteten Resourcen: Kontakte, Kalender-Events, Todos, Journal-Einträge, Free-Busy, \ldots
  \item Synchronisation von Collections für Offline-Nutzung
  \item einfach zu implementieren (vgl. CalDAV, CardDAV, IMAP) $\rightarrow$ ReST
  \item Standardkonform $\rightarrow$ xCard, xCal, ATOM
  \item nutzbar durch JavaScript $\leadsto$ JSON basierte Medientypen
  \end{itemize}
\end{frame}

\begin{frame}{Resourcen}
  \begin{itemize}
  \item Groupware Elemente: Kontakte, Kalender-Events, Todos, Journal-Einträge, Free-Busy
  \item Collections dieser Elemente
  \item Einstiegspunkt zum Auffinden der Collections
  \end{itemize}
\end{frame}

\begin{frame}{Medientypen}
  \begin{itemize}
  \item   application/vcard+xml
  \item   application/calendar+xml; component="VEVENT"
  \item   application/calendar+xml; component="VTODO"
  \item   application/calendar+xml; component="VJOURNAL"
  \item   application/calendar+xml; component="VFREEBUSY"
  \end{itemize}
  + ATOM(Pub) für Collections, Sevice Documents

  + (X)HTML ( +  Microformats ). Unterscheidung verschiedener HTML Varianten (view/edit)?

  + JSON Alternativen, z.B. collection+json, portable contacts? 
\end{frame}

\begin{frame}[fragile]{ATOM Service Document (application/atomsvc+xml)} 
  \begin{lstlisting}[language=xml]
<service xml:base="http://example.org"
  <workspace> <atom:title>Content Management
    <collection href="blog/main" >
      <atom:title>My Blog Entries
      <categories href="cats/forMain.cats">
    <collection href="gallery" >
      <atom:title>Pictures
      <accept>image/png <accept>image/jpeg

  <workspace> <atom:title>Groupware
    <collection href="groupware/contacts" >
      <atom:title>Kontakte
      <accept>application/vcard+xml

  \end{lstlisting}
\end{frame}

\begin{frame}{Synchronisierung}
  Feed Paging and Archiving, RFC 5005

  The Atom "deleted-entry" Element, draft-snell-atompub-tombstones-13

  Achtung: Server muß IDs gelöschter Einträge erinnern
\end{frame}

\section{Hypermedia Unterstützung}

\begin{frame}{Hyperlinks}
  \usetikzlibrary{positioning,arrows,fit}
  \usetikzlibrary{decorations.pathmorphing,backgrounds,fit,decorations.pathreplacing}
  \begin{center}
  \begin{tikzpicture}
    [doc/.style={rectangle,draw=blue!30,fill=blue!20, node distance=6em}]
    \node[doc] (svc) [] {ServiceDoc};
    \node[doc] (collect) [below=of svc,text width=5em] {Contacts Collection}
      edge [<-] node {} (svc)
      edge [->, loop left] node {next} (collect);

    \node[doc] (calcollect) [below right=of svc,text width=5em] {Calendar Collection}
      edge [<-] node {} (svc);

    \node[doc] (entry) [below=of collect] {Entry (xCard)}
      edge [<-] node {1..n} (collect)
      edge [->] node [text width=5em] {personal calendar} (calcollect)
      edge [->, loop left] node {related xCards} (entry);

    \node[doc] (event) [below=of calcollect] {Entry (xCal event)}
      edge [<-] node {1..n} (calcollect)
      edge [->] node [below right] {participants} (entry);


  \end{tikzpicture}
  \end{center}
\end{frame}

\section{Wiederverwendbare Komponenten}
\begin{frame}[fragile]{REST/ATOM framework}
  Jersey (JAX-RS): zu allgemein, ungeeignetes Programmiermodell
  \begin{lstlisting}
@PATH("/{USER}/collections/{COLLECTION}/{ENTRY}")
class EntryResource {
  @PathParam{"USER"} String user;
  @PathParam{"COLLECTION"} String collection;
  @PathParam{"ENTRY"} String entry;
  
  @GET @PROVIDES("application/vcard+xml")
  public VCard getVCard() {
  \end{lstlisting}
\end{frame}

\begin{frame}[fragile]{Warum nicht Jersey (JAX-RS)?}
\begin{itemize}
\item JAX-RS: Resource Klassen definieren ihre Pfade
  \begin{itemize}
  \item Kopplung Resource zu Ort
  \item Code Kopien: \verb:/{AUTHORITY}/{COLLECTION}/{ENTRY}:
  \item Deklaratives Linkbuilding (@REF annotation) \lightning sub-resources
  \end{itemize}
\item 3 Parameter für dispatch: HTTP Verb, Medientyp, Pfad -- nicht unabhängig behandelbar
\item schwer, Fehler nachzuvollziehen
\item Konflikt zwischen Jersey und DI Container: Guice, Spring
\item JAX-RS kennt keine Collections, Entries, Service documents
\item JAX-RS kennt keine Semantik für GET, POST, PUT, DELETE
\end{itemize}
Alternative: Apache Abdera
\end{frame}

\begin{frame}{Apache Abdera / ATOM}
  Abdera / ATOM kennt Resourcentypen: Collections, Entries, Service Documents, Categories

  Abdera versteht GET, POST, PUT, DELETE, HEAD, OPTIONS

  zu Implementieren: CollectionAdapter, WrappedEntry
\end{frame}

\begin{frame}[fragile]{generische Eigenschaften: interface WrappedEntry<T>}
  \begin{lstlisting}
    String getId() // eindeutig, konstant
    String getName() // URI Bestandteil
    String getTitle()
    DateTime getUpdated()
    Text getSummary(Factory, RequestContext)
    boolean isMediaEntry()
    Set<Person> getAuthors(Factory, RequestContext)
    Content getContent(Factory, RequestContext)
    T getWrappedEntry()
  \end{lstlisting}
\end{frame}

\begin{frame}{followup Silvia}
  ATOM bietet Werkzeugkasten von nützlichen Resourcetypen  

  DSL zur Beschreibung von Rest Anwendungen?
\end{frame}

\section{Herausforderungen}

\begin{frame}{Herausforderung: resource identifier}
  unveränderlich, eindeutig, sinnvoll
  \begin{itemize}
  \item id(NAME) -- veränderlich, nicht eindeutig
  \item UUID -- eindeutig, unveränderlich
  \item user input -- unpraktisch, eindeutig?
  \end{itemize}
  Lösung: FULLNAME-UUID, EVENTTITLE-UUID

  optional: HTTP redirect falls sich FULLNAME, EVENTTITLE ändert
\end{frame}

\begin{frame}{Medientypen nicht isomorph}
  Portable Contacts vs. xCard

  Google Calendar (Json) vs. xCal

  \begin{itemize}
  \item Nur einen kanonischen Medientypen für Updates zulassen? PATCH Verb?
  \item Erweiterbarkeit von Medientypen nutzen (X-\ldots Elemente)
  \end{itemize}
\end{frame}

\begin{frame}{Komplexität der Medientypen}
  \begin{itemize}
  \item eigenes, textbasiertes Serialisierungsformat (vCard)
  \item 12 data types (text, date, date-time, int, language-tag, \ldots)
  \item 11 property parameters: (lang, media type, sort-as, \ldots)
  \item 36 properties mit erlaubten property parameters, data types
  \item + Erweiterungen
  \end{itemize}
  Validierung eingehender Resourcen?

  \ldots keine brauchbare Java Bibliothek
\end{frame}
\end{document}
