\documentclass[12pt,a4paper]{scrartcl}		% KOMA-Klassen benutzen!

%\usepackage[ngerman]{babel}			% deutsche Namen/Umlaute
\usepackage[utf8]{inputenc}			% Zeichensatzkodierung
\usepackage{url}
\usepackage{graphicx}
\usepackage[colorlinks=false, pdfborder={0 0 0}]{hyperref}
\usepackage{amsmath}
\usepackage{multicol}
\usepackage{glossaries}

\usepackage{setspace} % Anderthalbfacher Zeilenabstand ist Standard in den meisten Seminararbeiten. Das Paket setspace ermöglicht ein einfaches Umstellen von normalem, anderthalbfachen oder sogar doppeltem Zeilenabstand. 
\usepackage[paper=a4paper,inner=25mm,outer=20mm,top=15mm,bottom=20mm]{geometry} %Das geometry Paket dient zur Einrichtung der Seiten. Hier werden die jeweiligen Seitenränder angegeben. Diese wWerte sollten durch die jeweiligen Vorgaben des Seminarleiters oder Instituts ersetzt werden.
\setlength{\parindent}{1.7em} %Neue Abschnitte werden mit hängendem Einzug gesetzt, parindent definiert. um wie viel der Absat eingerückt wird. Die Einheit em ist abhängig vom verwendeten Zeichensatz und daher absoluten Werten in mm oder cm vorzuziehen. 
\setcounter{secnumdepth}{3} %Bis zu welcher Gliederungsebene nummeriert werden soll gibt dieser Befehl vor. In diesem Falle werden \section, \subsection und \subsubsection nummereiert.
\setcounter{tocdepth}{3} %Bis zu welcher Ebene Einträge ins Inhaltsverzeichnis aufgenommen werden. In diesem Beispiel ebenfalls bis Ebene drei (\subsubsection). Ein durch \paragraph ausgewiesener Abschnitt wird demnach nicht im Inhaltsverzeichnis auftauchen. 

\newcommand{\citeurl}[2]{\url{#1} (#2)}

\begin{document}
%\titlehead{}
%\subject{subject}
\title{}
\subtitle{}
\author{Thomas Koch\\\url{thomas@koch.ro}\\matriculation number 7250371}
\publishers{Fernuniversität Hagen\\Faculty of mathematics and computer science}
\date{\today}
%\thanks{}
\maketitle{}

%\newpage{}
\tableofcontents{}
\begin{abstract}
  This report documents my work done and observations made in the Apache ZooKeeper project since January 2010 for recognition as an advanced internship module.
\end{abstract}
\newpage{}

\appendix

\section{Standards}

\subparagraph{IETF (RFC)}
\begin{description}

  \item[RFC 5545 Internet Calendaring and Scheduling Core Object Specification (iCalendar)]

    iCalendar is the core data schema for calendaring information. This is the
    new version and obsoletes RFC2445

  \item[RFC 6321 xCal: The XML format for iCalendar]

    This specification defines a format for representing iCalendar data in
    XML. More specifically, is to define an XML format that allows iCalendar
    data to be converted to XML, and then back to iCalendar, without losing any
    semantic meaning in the data. Anyone creating XML calendar data according to
    this specification will know that their data can be converted to a valid
    iCalendar representation as well.

\end{description}
\subparagraph{W3C}
\subparagraph{others}
\subparagraph{}

\section{People, Groups and Organizations}

Eliot Lear <lear@cisco.com>
  IETF Calsify WG chair

Lisa Dusseault is a development manager and standards architect at the Open Source Applications Foundation, where she's involved in the Chandler, Cosmo and Scooby projects. Previously, Lisa came from Xythos, an Internet startup where she was development manager for four years. She has also been an IETF contributor on various Internet applications protocols for eight years now, and continues to do this kind of work at OSAF. She co-chairs the IETF IMAP extensions and CALSIFY (Calendaring and Scheduling Standards Simplification) Working Groups. She is also the author of a book on WebDAV and co-author of CalDAV, an open and interoperable protocol for calendar access and sharing. 

Peter Saint-Andre <stpeter@stpeter.im>
  IETF Calsify WG area director

Joseph Smarr
  former Plaxo now Google
  presentation about portable contacts at vcarddav wg http://tools.ietf.org/agenda/74/slides/vcarddav-2.pdf
  http://josephsmarr.com
  http://anyasq.com/79-im-a-technical-lead-on-the-google+-team

\section{Implementations}



% http://en.wikibooks.org/wiki/LaTeX/Glossary
\bibliography{references}{}
\bibliographystyle{alphadin}
\end{document}

% Local Variables:
% ispell-dictionary: "american"
% eval: (flyspell-mode 1)
% End: