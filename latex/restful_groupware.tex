\documentclass[12pt,a4paper]{scrartcl}		% KOMA-Klassen benutzen!

%\usepackage[ngerman]{babel}			% deutsche Namen/Umlaute
\usepackage[utf8]{inputenc}			% Zeichensatzkodierung
\usepackage{url}
\usepackage{graphicx}
\usepackage{amsmath}
\usepackage{multicol}
\usepackage{glossaries}
\usepackage{expdlist}
\usepackage{subfig}
\usepackage{listings}

\usepackage{setspace} % Anderthalbfacher Zeilenabstand ist Standard in den meisten Seminararbeiten. Das Paket setspace ermöglicht ein einfaches Umstellen von normalem, anderthalbfachen oder sogar doppeltem Zeilenabstand. 
\usepackage[paper=a4paper,inner=25mm,outer=20mm,top=15mm,bottom=20mm]{geometry} %Das geometry Paket dient zur Einrichtung der Seiten. Hier werden die jeweiligen Seitenränder angegeben. Diese wWerte sollten durch die jeweiligen Vorgaben des Seminarleiters oder Instituts ersetzt werden.
\setlength{\parindent}{1.7em} %Neue Abschnitte werden mit hängendem Einzug gesetzt, parindent definiert. um wie viel der Absat eingerückt wird. Die Einheit em ist abhängig vom verwendeten Zeichensatz und daher absoluten Werten in mm oder cm vorzuziehen. 
\setcounter{secnumdepth}{3} %Bis zu welcher Gliederungsebene nummeriert werden soll gibt dieser Befehl vor. In diesem Falle werden \section, \subsection und \subsubsection nummereiert.
\setcounter{tocdepth}{3} %Bis zu welcher Ebene Einträge ins Inhaltsverzeichnis aufgenommen werden. In diesem Beispiel ebenfalls bis Ebene drei (\subsubsection). Ein durch \paragraph ausgewiesener Abschnitt wird demnach nicht im Inhaltsverzeichnis auftauchen. 

\usepackage[colorlinks=false, pdfborder={0 0 0}, plainpages=false]{hyperref}

\newcommand{\citeurl}[2]{\url{#1} (#2)}

\begin{document}
%\titlehead{}
%\subject{subject}
\title{}
\subtitle{}
\author{Thomas Koch\\\url{thomas@koch.ro}\\matriculation number 7250371}
\publishers{Fernuniversität Hagen\\Faculty of mathematics and computer science}
\date{\today}
%\thanks{}
\maketitle{}

%\newpage{}
\tableofcontents{}
\begin{abstract}

\end{abstract}
\newpage{}

\section{Motivation}

Experiences with Android and Data in the cloud
\url{http://keithp.com/blogs/calypso}

Why vCard/CardDav: many clients

Why OpenSocial / Portable Contacts:
\begin{itemize}
\item used by Google, LinkedIn,
\item used in Enterprise applications like Attlassian tools (Jira, Confluence, ...), Nuxeo CMS, ...
\item OpenSocial can be used to implement inhouse portals and populate it with data from the companies GroupWare
\end{itemize}


\section{Advantages of REST for Groupware data}

The REST style is motivated by the argument that its application would help to provide certain attributes for an architecture.\cite[sec 5.1]{Fielding2000}

Maybe not so important for a companies internal Groupware: Scalability (in terms of users), Network performance, Efficiency, 

\begin{itemize}
\item Cachability can keep the data available also in offline mode
\item Simplicity helps to develop glue code between different systems
\item Modifiability allows to adapt the Groupware to changes in the organization
\item Reliability
\item Administrativ scalability
\end{itemize}

\section{Evaluation of APIs}

\subsubsection{Algermissen's Classification of HTTP-based APIs}
\label{sec:algerm-class-http}

Jan Algermissen, proposes a ``Classification of HTTP-based APIs'' in February 2010.\footnote{
\citeurl{http://nordsc.com/ext/classification_of_http_based_apis.html}{2011-12-08}} 
He identifies and names a five level order of HTTP based APIs. Each level adds
one constraint to be obeyed. The last level obeys all five constraints and is
RESTful.~(\autoref{tab:algermissenclassification})

\begin{table}[h]
  \begin{tabular}{p{0.15\textwidth} p{0.3\textwidth} p{0.4\textwidth}}
    level name & constraint violated & violation description \\
    \hline \\
    WS-* &   Identification of Resources & Only service endpoint is identified by URI. No resources exposed. \\
    RPC URI-Tunneling &    Manipulation of Resources through Representations & SOAP body contains operation name, message not transferred to manipulate resource state. \\
    HTTP-based Type I &    Self-Descriptive Messages & Message semantics depend on action specified in message body. \\
    HTTP-based Type II &    Hypermedia as the Engine of Application State & Application state machine known at design time. \\
  \end{tabular}
  \caption{Classification of HTTP-based APIs after Algermissen}
  \label{tab:algermissenclassification}
\end{table}

In addition to that, Algermissen also provides a list of ``Effect on System
Properties and Cost'' of the different API styles and acknowledges that the
initial costs of developing a REST API may be higher compared to the other
styles but lower in the long run.



\section{Existing Groupware data APIs}

\subsection{IMAP (Kolab)}

Kolab uses an IMAP server as the data store and
synchronization protocol for calendar and contact informations. I want to
compare this approach to a restful one.

Advantages of IMAP:
\begin{itemize}
\item already there, since Mail uses it
\item can store blobs/files so no need to map the iCal/vCard files to a relational scheme
\item out of the box support for offline work and later synchronization (How does it solve editing conflicts?)
\end{itemize}
Disadvantages of IMAP:
\begin{itemize}
\item Complicate, 38 RFCs according to \url{http://de.wikipedia.org/wiki/Internet_Message_Access_Protocol} see also: \url{http://www.apps.ietf.org/rfc/ipoplist.html}
\item All clients directly access the iCal/vCard files with no moderation layer in between. This means that no validation or normalization can be done. Schema updates can only be done if all clients cooperate.
\item IMAP imposes a folder structure. Google's gmail is an example for another, tag based approach. Messages could have several tags. It is therefor hard to access Gmail via IMAP.
\item Sam Varshavchik, the author of the courier Mail Transfer Agent explains the history of IMAP and claims that the IMAP standard is broken: http://www.courier-mta.org/fud/
\item IMAP is so complicate that the IMAP wiki holds 10 pages of advises for IMAP client authors: http://www.imapwiki.org/ClientImplementation RFC 2683 ``IMAP4 Implementation Recommendations'' is a 23 pages document (cut 5 pages for verbosity) explaining how to implement another RFC standard. Is there any widely used standard that needs another RFC explaining how to implement it?
\item \url{http://en.wikipedia.org/wiki/Internet_Message_Access_Protocol#Disadvantages}
\item Some attempts to create a simpler alternative to IMAP:
  \begin{itemize}
  \item http://en.wikipedia.org/wiki/POP4
  \item \url{http://en.wikipedia.org/wiki/Simple_Mail_Access_Protocol} also here http://www.courier-mta.org/cone/smap1.html
  \item \url{http://en.wikipedia.org/wiki/Internet_Mail_2000}
  \item HTTP restful: http://tools.ietf.org/id/draft-dusseault-httpmail-00.txt mailing list: https://www.ietf.org/mailman/listinfo/httpmail
  \item BikINI is not IMAP http://bikini.caterva.org
  \item Outlook uses HTTP to communicate with Hotmail
  \item another rest mail proposal: http://www.prescod.net/rest/restmail/
  \end{itemize}
\item more rants: http://blog.gaborcselle.com/2010/02/how-to-replace-imap.html
\item IMAP issues found by the chandler project http://chandlerproject.org/bin/view/Jungle/IntrinsicIMAPIssues
\end{itemize}

\subsection{WebDAV, CalDAV, CardDAV}

\subsubsection{Is WebDAV restful?}

Roy Fielding says no: \url{http://tech.groups.yahoo.com/group/rest-discuss/message/5874}

\begin{quotation}
PROP* methods conflict with REST because they prevent
important resources from having URIs and effectively double the
number of methods for no good reason. Both Henrik and I argued
against those methods at the time. It really doesn't matter
how uniform they are because they break other aspects of the
overall model, leading to further complications in versioning
(WebDAV versioning is hopelessly complicated), access control
(WebDAV ACLs are completely wrong for HTTP), and just about every
other extension to WebDAV that has been proposed.

[\ldots]

The problem with MOVE is that it is actually an operation on two
independent namespaces (the source collection and destination
collection). The user must have permission to remove from the
source collection and add to the destination collection, which
can be a bit of a problem if they are in different authentication
realms. COPY has a similar problem, but at least in that case
only one namespace is modified. I don't think either of them map
very well to HTTP.
\end{quotation}

The discussion also continued on the microformats mailing list
\url{http://microformats.org/discuss/mail/microformats-rest/2006-April/thread.html#217}.

see \cite{Amundsen2010} for a restful approach to properties.

Is ATOM an alternative to WebDAV?

\begin{quote}
  AtomPub is different from DAV in two key respects:
  \begin{itemize}
  \item The client doesn't control where things go, the server does
  \item It is allowed and expected that an AtomPub server will look at the incoming information and change it (generate ID, timestamps, sanitize HTML, etc)
  \end{itemize}
\end{quote}
Tim Bray, http://www.imc.org/atom-protocol/mail-archive/msg11271.html


\subsection{PortableContacts aka OpenSocial}

\subsubsection{Is OpenSocial restful?}

Roy Fielding wrote a blog post about the ``SocialSite REST API'', stating that
it isn't restful at all but clearly an RPC style
API.\footnote{\citeurl{http://roy.gbiv.com/untangled/2008/rest-apis-must-be-hyperhtext-driven}{2011-12-06}}
Fielding was referring to SocialSite, which is however an implementation of the
OpenSocial specification. Dave Johnson, a contributor to SocialSite, reacted on
this critique by opening a discussion on an OpenSocial mailing
list:\footnote{\citeurl{http://groups.google.com/group/opensocial-and-gadgets-spec/browse_thread/thread/aff4ba7373e21284/201a413efa67c26e}{2011-12-06}}
\begin{quote}
  I must admit, it is not clear to me how OpenSocial REST API violates the six
  rules that Roy stated.
\end{quote}
The above quote warrants a short comment. I also thought before, that REST would
be so simple that there wouldn't be much need for further studying. Every web
developer has some understanding of URIs, HTTP and a bit less of Hypermedia. So
it is easy to fall into the trap that everything build on top of HTTP would be
restful. Now however, after some more reading about REST, I can easily find
violations of the REST constraints in the OpenSocial specification.

Restful APIs are modeled around resources, their representations and links
between them. The authors of the OpenSocial API however seem to have modeled
their API around the concept of services:\cite[Social API Server, sec 2, Services]{OSSpec2.0.1}
\begin{quote}
  OpenSocial defines several services for providing access to a container's data.
\end{quote}

\paragraph{Fielding's critique}
Fielding listed some rules that a restful API must obey, but did not give
explicit examples how OpenSocial violates this rules. The following section will
provide such examples.

\begin{quote}
  A REST API should not be dependent on any single communication protocol,
  [\ldots] any protocol element that uses a URI for identification must allow
  any URI scheme to be used for the sake of that identification.
  \textit{[Failure here implies that identification is not separated from
    interaction.]}
\end{quote}

OpenSocial defines a construct called ``REST-URI-Fragment'' which is a clear
violation of the above rule. This URI fragment is simply an encoding of
procedural parameters as elements of an HTTP URI:\cite[Core API Server, sec
2.1.1.2.2, REST-URI-Fragment]{OSSpec2.0.1}

\begin{quote}
  Each service type defines an associated partial URI format. The base URI for
  each service is found in the URI element associated with the service in the
  discovery document. Each service type accepts parameters via the URL
  path. Definitions are of the form:
  
  \verb:{a}/{b}/{c}:
\end{quote}

URIs can contain a query component that would be more appropriate to contain
parameters. This would also have made it clearer to see that the specification
actually defines services instead of resources. One test showing the misfit is
to ask how dot-segments ('.' and '..') inside the URI fragment are interpreted
and whether this conforms with the letter and spirit of the URI
standard.\cite[sec 3.3]{RFC3986} Another misfit can be seen in the URI fragment
to retrieve one or multiple albums. In this case the 'c' part in the quoted
definition above is actually a list of albums to retrieve separated by a slash.

Fielding's second bullet point most likely refers to the
\texttt{X-HTTP-Method-Override} header. This header is a widely
used\footnote{\citeurl{http://www.subbu.org/blog/2008/07/another-rest-anti-pattern}{2011-12-06}}
workaround to allow the use of other HTTP methods than GET and POST from HTML
forms or through firewalls.

The next two points again refer to a more serious issue:

\begin{quote}
  A REST API should spend almost all of its descriptive effort in defining the
  media type(s) used for representing resources and driving application
  state[\ldots].  \textit{[Failure here implies that out-of-band information is
    driving interaction instead of hypertext.]}  A REST API must not define
  fixed resource names or hierarchies[\ldots] \textit{[Failure here implies that
    clients are assuming a resource structure due to out-of band
    information[\ldots]].}
\end{quote}

The OpenSocial specification contains a lot of out-of-band information
describing how to form URIs to access information or which methods to use on
which URIs for different actions. This means that the OpenSocial API is not
simple or intuitive to use but requires a client developer to read a lot of
specification, thus violating the simplicity property of a restful
architecture. Since the URIs are fixed in the specification and necessarily also
in clients, the modifiability property is also violated.\cite[sec
2.3]{Fielding2000}

The following tables give some examples of the specified URIs:

\begin{table}[h]
\begin{tabular}{p{6.5cm} l p{10cm}}
  URI fragment & Method & Description \\
  %\hline
  \verb:/people/{User-Id}/@self: & GET & profile for User-Id \\
  \verb:/people/{User-Id}/@self: & DELETE & remove User \\
  \verb:/people/{User-Id}/{Group-Id}: & GET & full profiles of group members \\
  \verb:: & POST & Create relationship, target specified \newline by \verb:<entry><id>: in body \\
   & POST & Update Person \\
  \verb:/people/{Initial-User-Id}/: \newline \verb:{Group-Id}/{Related-User-Id}: & GET & ??? \\
  \verb:/people/@supportedFields: & GET & list of supported person profile fields \\
  \verb:/groups/{User-Id}[/{Group-Id}]: & GET & one or all groups of a user \\
   & PUT & update group \\
   & DELETE & delete group \\
  \verb:/groups/{User-Id}: & POST & create group \\
\end{tabular}
  \caption{URI fragments for peoples and groups in the OpenSocial REST API}
\end{table}

\begin{table}[h]
\begin{tabular}{p{9.5cm} l p{8cm}}
  URI fragment & Method & Description \\
  %\hline
  \verb:/albums/{User-Id}/@self: & POST & create album \\
  \verb:/albums/{User-Id}/{Group-Id}[/Album-Id]*: & GET & one or multiple albums \\
  \verb:/mediaItems/{User-Id}/{Group-Id}/{Album-Id}/: \newline \verb:{MediaItem-Id}: & GET & one mediaitem \\
  \verb:/mediaItem/{User-Id}/@self/{Album-Id}: (sic!) & POST & create mediaitem \\
\end{tabular}
  \caption{URI fragments for albums and mediaitems in the OpenSocial REST API}
  \label{tab:OSURIAlbums}
\end{table}

The last URI in \autoref{tab:OSURIAlbums} is obviously missing an ``s'' behind
\texttt{mediaItem}. This typo is present and unfixed in the OpenSocial spec
since Version 1.0, released in march 2010. This is of course not a big issue in
itself, but rather a sign that the specification is too verbose and does
over-specify things that should rather be auto-discovered through hyperlinks.

Fielding mentions in a comment to the same blog post that the OpenSocial API
``could be made so [restful] with some relatively small changes'' but does not
specify these changes. However some issues can easily be identified.

First the data structures defined in OpenSocial do not use URIs to refer to
other resources. Instead they use Object-Ids that must then be inserted in the
appropriate URI templates. Examples are the \texttt{recipients},
\texttt{senderId}, \texttt{collectionIds} of messages and the \texttt{ownerId}
of albums. The person structure does not contain fields referencing other
resources. Thus it does not obviously violate REST like the albums and
messages. However it does so even worse since there are hidden references only
defined out-of-band in the specification. One can retrieve the albums, relations
or messages of a user by filling in the \texttt{userId} in one of the specified
URI templates. If Users would just contain references to other resources related
to a user, the specification could already be shortened a lot.

Another missed opportunity for a much more intuitive API is the relation of
media items and albums. This seems to be poster child example for a collection
(album) to collection-element (media item) relation which could have made use of
the hierarchical character of URI paths. OpenSocial however requires the client
developer to use two different URI templates. (\autoref{tab:OSURIAlbums})

A not so small change to OpenSocial would be to either use already standardized
and registered media types where possible or to register new types where
necessary. It seems that there are some already existing media types that could
be a good fit for OpenSocial but only miss a canonical json representation for
easy consumption by javascript applications. These are vCard for
persons,\footnote{OpenSocial persons are based on portable contacts which in
  turn borrowed field names from vCards.} ATOM entries\cite{RFC4287} for
messages, activities and media items and ATOM categories, collections or
workspaces\cite{RFC5023} for albums and groups. It would probably be necessary
to add extensions to the mentioned media types but vCard and ATOM both already
anticipated this need and provided mechanisms to do so.

The use of the ATOM format could promote the adoption of OpenSocial because
developers could either reuse existing knowledge about ATOM or would be more
motivated to learn about a system that is based on an already widespread
format. In fact OpenSocial already mentions ATOM as a way to wrap OpenSocial
data. However this wrapping does not build extend and reuse ATOM semantics as
proposed above but just puts the OpenSocial data structures inside the
entry/content element of ATOM. This kind of misuse of ATOM does of course not
deliver any advantage on top of the existing plain JSON or XML
representations.\footnote{compare Bill de Hora, Extensions v Envelopes. 11/2009 
%\newline
  \citeurl{http://www.dehora.net/journal/2009/11/28/extensions-v-envelopes}{2011-21-07}}
Consequently the newest OpenSocial specification deprecates any reference to the
ATOM format.

In Algermissen's classification (~\ref{sec:algerm-class-http}), the OpenSocial
REST API would actually be ``HTTP-based Type I'' due to the lack of media types
and direct hyper links between related resources. Algermissen writes that this
level has the lowest possible initial cost of all HTTP APIs. Or in other words:
The OpenSocial specification authors did not have to invest a lot to come up
with this API specification but maintenance and evolution cost may be medium or
high.


\section{ATOM}

Google's Data API extends ATOM?
Calendar feeds are available as ATOM with the published entry set to the date of the event.

WebDAV vs. ATOM:
\url{http://intertwingly.net/wiki/pie/WebDav}
\url{http://intertwingly.net/wiki/pie/WebDavVsAtom}
google webdav atom

\section{Persistency for Groupware Data}
Relational Databases vs. NoSQL databases vs. plain files

Relational databases are not practical for contacts, events or todos. Common patterns in systems that use relational DBs for that purpose:
\begin{itemize}
\item artificial limits of entries, e.g. only 3 email addresses per contact, because there are only three columns email1, email2 and email3.
\item Fields for custom data like custom1 to customX
\item EAV pattern: tables like: id, foreign\_id, type, value
\end{itemize}
\section{Synchronizing a large collection}

How to efficiently synchronize a large collection of contacts with the server without checking each contact for changes?

Portable Contacts has a filter ``updatedSince''.

How is synchronization done in CardDAV?

\section{Mediatypes}

\begin{quote}
  To some extent, people get REST wrong because I failed to include enough
  detail on media type design within my dissertation.~--~Roy T. Fielding
\end{quote}
from: Rest APIs must be hypertext driven\footnote
  {\citeurl{http://roy.gbiv.com/untangled/2008/rest-apis-must-be-hypertext-driven}{2011-12-20}}

\subsection{Syntax vs. Semantic}

The usage of standardized media types is one key difference between an API and a
restful API. \cite[sec. 5.2.1.2]{Fielding2000} Only if the client has knowledge
about the media type can it do something meaningful with it besides just
receiving it. In that sense, the often used mime types application/xml or
application/json are not really media types. They don't tell the browser or user
anything meaningful beside the \emph{syntax} of the data.\footnote{
\citeurl{http://blog.programmableweb.com/2011/11/18/rest-api-design-putting-the-type-in-content-type}{2011-21-20}
and Web Resource Modeling Language \citeurl{http://www.wrml.org}{2011-12-20} both by Mark Massé
}

To do anything meaningful with plain json or xml, the client programmer must
normally look up the meaning or \emph{semantic} of the data in the API
documentation. The data therefor fails the self-descriptive constraint of
REST.

Compare this with a mime type like \texttt{application/atom+xml}. It specifies
the syntax (xml) and the semantic (atom) of the data. Of course somebody once
needed to read the atom specification and program the client with the knowledge
of how to process this media type. The purpose of standardized media types
however is that there is a limited number of them and that they are reused by many
sites and clients.

Large sites like Google, Facebook or Twitter can successfully attract enough
developers to read their specifications and program clients accordingly. Usually
one can also find client libraries that can help a lot. REST however envisions a
decentralized web in which parties can interact without previous knowledge of
each other. This becomes possible through the usage of well known predefined
media types.

\subsection{xml vs. json}

The application section of the IANA mime type registration has 294 entries
ending in ``+xml'' and only 3 ending in
``+json''.\footnote{\citeurl{http://www.iana.org/assignments/media-types/application/index.html}{2011-12-20}}
This stands in contrast to the rise of public json APIs and the decline of XML
APIs.\footnote{\citeurl{http://blog.programmableweb.com/2011/05/25/1-in-5-apis-say-bye-xml/}{2011-12-20} \citeurl{http://www.readwriteweb.com/cloud/2011/03/programmable-web-apis-popping.php}{2011-12-21}}

JSON is often preferred as a format over XML because it is perceived as easier
to parse, process and smaller in size. XML in comparison is seen as complicate,
slow to process and larger in size. A strong argument for JSON as the preferred
format for JavaScript applications is that JSON is a subset of JavaScript.

A drawback of this mismatch between the preference of media type designers and
API consumers is a possible duplication of work and incompatibilities across
different APIs. If two developers independently take the same XML media type as
a model to develop a JSON format, they can still come up with different
results. There is no standard mapping from XML schemes to JSON structures.

Instead, possible mappings have to trade of the preservation of all structural
information against the ``friendliness'' of the resulting JSON
structure.\cite{Boyer2011} Without going into detail, a JSON structure can be
seen as friendly if it makes best use of JSON's data types, is compact and easy
to process. \autoref{fig:waysmapxmljson} shows two different examples how to map
data from XML to JSON with one of them using JSON number values, being more
compact and probably easier to process.

\begin{figure}
\captionsetup[subfloat]{justification=raggedright,singlelinecheck=false}
\small{
\subfloat[XML fragment]{
  \begin{minipage}[b]{0.25\linewidth}
    \lstinputlisting{snippets/xml2json_xml}    
  \end{minipage}
}
\subfloat[unfriendly JSON]{
  \begin{minipage}[b]{0.4\linewidth}
    \lstinputlisting{snippets/xml2json_json1}
  \end{minipage}
}
\subfloat[less(?) unfriendly JSON]{
  \begin{minipage}[b]{0.25\linewidth}
    \lstinputlisting{snippets/xml2json_json2}
  \end{minipage}
}
}
  \caption{different ways to map XML to JSON}
  \label{fig:waysmapxmljson}
\end{figure}



 in the area of semantic
standardization. Questions whether and how contact information should also hold
space for the place of birth and place and date of death of a
person\footnote{\citeurl{https://datatracker.ietf.org/doc/draft-ietf-vcarddav-birth-death-extensions/}{2011-12-20}}
are independent from a data serialization format. So is the question whether and
how to specify the sex or gender of a
person.\footnote{\citeurl{http://www.ietf.org/mail-archive/web/vcarddav/current/msg01778.html}{2011-12-20}}





Version 3 of vCard was published in 1998\cite{RFC2425} only a few months after
the W3C published Version 1.0 of XML\cite{Paoli:98:XR} and eight years before
JSON became an official standard.\cite{RFC4627}

XML standards:
  xCal, xCard, XMPP, Atom

\url{http://en.wikipedia.org/wiki/ECMAScript_for_XML} (E4X) makes XML a first class language construct 

\begin{table}
  \begin{tabular}{l c c c c c}
    type of data & XML  & JSON                      & semantic          & microformat & comment \\
    Calendar     & xCal & Google calendar API       &                   & hCalendar & other: iCalendar  \\
    Contact      & xCard & portable contacts, jCard & friend of a friend & hCard & other: vCard \\
    Resume       & HR XML &                          & Description of a Career & hResume & \\    
  \end{tabular}
  \caption{data in different formats}
  \label{tab:data-formats}
\end{table}

Open-Xchange provides an (unrestful) HTTP/JSON API which is used by its
javascript
frontend.\footnote{\citeurl{http://oxpedia.org/index.php?title=HTTP_API}{2011-19-12}}
The comprehensive documentation does not indicate whether the data structures
for tasks, appointments, reminders and contacts were inspired by any
standards. In any case the API documentation is a good example of the need for
standard mime types in JSON format. The use case for this API also shows
similarities to the use of OpenSocial for intranet frontends.

\subsection{Updates with alternative Mediatypes}
How to handle updates, if the mediatypes are not isomorph?

How does Google handle PATCH in the calendar API?

\subsection{Mediatype conversion}

Which fields of portable contacts are derived from vCard: \url{http://wiki.portablecontacts.net/w/page/17776141/schema}

\begin{quote}
  No single data representation is ideal for every client. This protocol defines representations for each resource in three widely supported formats, JSON [RFC4627], XML, and Atom [RFC4287] / AtomPub [RFC5023], using a set of generic mapping rules. The mapping rules allow a server to write to a single interface rather than implementing the protocol three times.
\end{quote}\cite[Core API Server]{OSSpec2.0.1}

% microformats to json converter \url{http://microformatique.com/optimus/}


In 2007, a project called microjson wanted to standardize json representations of microformat data structures.\footnote{\citeurl{http://notizblog.org/2007/09/16/microjson-microformats-in-json/}{2011-12-19}} 

The project identified the need for a json schema:\footnote{\citeurl{http://web.archive.org/web/20080524003749/http://microjson.org/wiki/Schemas}{2022-12-19}}
\begin{quote}
  If there are standard microJSON formats for transfer of certain datasets, there will be a need to validate that data to ensure that it is infact valid format. To validate a format you need something that details the structure, data content types and required data. Sounds like we'll be needing a schema for each microJSON format. 
\end{quote}


jCard example from microjson.org\footnote{\citeurl{http://web.archive.org/web/20080517003233/http://microjson.org/wiki/JCard}{2011-12-19}}
\begin{lstlisting}
{
"vcard":{
  "name":{
    "given":"John",
    "additional":"Paul",
    "family":"Smith"
  },
  "org":"Company Corp",
  "email":"john@companycorp.com",
  "address":{
    "street":"50 Main Street",
    "locality":"Cityville",
    "region":"Stateshire",
    "postalCode":"1234abc",
    "country":"Someplace"
  },
  "tel":"111-222-333",
  "aim":"johnsmith",
  "yim":"smithjohn"
}
\end{lstlisting}

\subsection{Example: vCard}

\begin{lstlisting}
   <?xml version="1.0" encoding="UTF-8"?>
   <vcards xmlns="urn:ietf:params:xml:ns:vcard-4.0">
     <vcard>
       <fn><text>Simon Perreault</text></fn>
       <n>
         <surname>Perreault</surname>
         <given>Simon</given>
         <additional/>
         <prefix/>
         <suffix>ing. jr</suffix>
         <suffix>M.Sc.</suffix>
       </n>
       <bday><date>--0203</date></bday>
       <anniversary>
         <date-time>20090808T1430-0500</date-time>
       </anniversary>
       <gender><sex>M</sex></gender>
       <lang>
         <parameters><pref><integer>1</integer></pref></parameters>
         <language-tag>fr</language-tag>
       </lang>
       <lang>
         <parameters><pref><integer>2</integer></pref></parameters>
         <language-tag>en</language-tag>
       </lang>
       <org>
         <parameters><type><text>work</text></type></parameters>
         <text>Viagenie</text>
       </org>
       <adr>
         <parameters>
           <type><text>work</text></type>
           <label><text>Simon Perreault
   2875 boul. Laurier, suite D2-630
   Quebec, QC, Canada
   G1V 2M2</text></label>
         </parameters>
         <pobox/>
         <ext/>
         <street>2875 boul. Laurier, suite D2-630</street>
         <locality>Quebec</locality>
         <region>QC</region>
         <code>G1V 2M2</code>
         <country>Canada</country>
       </adr>
       <tel>
         <parameters>
           <type>
             <text>work</text>
             <text>voice</text>
           </type>
         </parameters>
         <uri>tel:+1-418-656-9254;ext=102</uri>
       </tel>
       <tel>
         <parameters>
           <type>
             <text>work</text>
             <text>text</text>
             <text>voice</text>
             <text>cell</text>
             <text>video</text>
           </type>
         </parameters>
         <uri>tel:+1-418-262-6501</uri>
       </tel>
       <email>
         <parameters><type><text>work</text></type></parameters>
         <text>simon.perreault@viagenie.ca</text>
       </email>
       <geo>
         <parameters><type><text>work</text></type></parameters>
         <uri>geo:46.766336,-71.28955</uri>
       </geo>
       <key>
         <parameters><type><text>work</text></type></parameters>
         <uri>http://www.viagenie.ca/simon.perreault/simon.asc</uri>
       </key>
       <tz><text>America/Montreal</text></tz>
       <url>
         <parameters><type><text>home</text></type></parameters>
         <uri>http://nomis80.org</uri>
       </url>
     </vcard>
   </vcards>
\end{lstlisting}

\begin{lstlisting}
   <?xml version="1.0" encoding="UTF-8"?>
   <vcards xmlns="urn:ietf:params:xml:ns:vcard-4.0">
     <vcard>
       <fn><text>Simon Perreault</text></fn>
       <n>
         <surname>Perreault</surname>
         <given>Simon</given>
         <suffix>ing. jr</suffix>
         <suffix>M.Sc.</suffix>
       </n>
       <bday day="02" month="03" />
       <anniversary format="date-time">20090808T1430-0500</anniversary>
       <gender>M</gender>
       <lang pref="1">fr</lang>
       <lang pref="2">en</lang>
       <org type="work">Viagenie</org>
       <adr type="work">
         <label>Simon Perreault
   2875 boul. Laurier, suite D2-630
   Quebec, QC, Canada
   G1V 2M2</label>
         <street>2875 boul. Laurier, suite D2-630</street>
         <locality>Quebec</locality>
         <region>QC</region>
         <code>G1V 2M2</code>
         <country>Canada</country>
       </adr>
       <tel>
         <type>work</type>
         <type>voice</type>
         <uri>tel:+1-418-656-9254;ext=102</uri>
       </tel>
       <tel>
         <type>work</type>
         <type>text</type>
         <type>voice</type>
         <type>cell</type>
         <type>video</type>
         <uri>tel:+1-418-262-6501</uri>
       </tel>
       <email type="work">simon.perreault@viagenie.ca</email>
       <geo type="work">
         <uri>geo:46.766336,-71.28955</uri>
       </geo>
       <key type="work">
         <uri>http://www.viagenie.ca/simon.perreault/simon.asc</uri>
       </key>
       <tz>America/Montreal</tz>
       <url type="home">
         <uri>http://nomis80.org</uri>
       </url>
     </vcard>
   </vcards>
\end{lstlisting}

\subsection{HFactor}
Mike Amundsen defines a method to asses media types that he calls
``HFactor''.\footnote{\citeurl{http://amundsen.com/hypermedia/}{2011-12-21}} The
HFactor distinguishes different types of support for links and indicates which
of those are provided by a reviewed media type.

Amundsen did reviews of a couple of media types. Unfortunately these do not
include \texttt{vcard+xml} or \texttt{calendar+xml}. I'll try to identify the
HFactors of both here.

The different types of link support have two letter acronyms and fall in two
categories: Link support values, with the first letter ``L'' and Control data
support, first letter ``C''.

\begin{itemize}
\item Link Support for
  \begin{itemize}
  \item \texttt{LE} embedded links (HTTP GET)
  \item \texttt{LO} out-bound navigational links (HTTP GET)
  \item \texttt{LT} templated queries (HTTP GET)
  \item \texttt{LN} non-idempotent updates (HTTP POST)
  \item \texttt{LI} idempotent updates (HTTP PUT, DELETE) 
  \end{itemize}
\item Control Data Support to
  \begin{itemize}
  \item \texttt{CR} modify control data for read requests (e.g. \texttt{HTTP Accept-*} headers)
  \item \texttt{CU} modify control data for update requests (e. g. \texttt{Content-*} headers)
  \item \texttt{CM} indicate the interface method for requests (e.g. HTTP GET,POST,PUT,DELETE methods)
  \item \texttt{CL} add semantic meaning to link elements using link relations (e.g. HTML rel attribute)
  \end{itemize}
\end{itemize}

\section{Hypermedia in RESTful applications}

% http://restpatterns.org/Articles/The_Hypermedia_Scale

% http://linkednotbound.net/2010/12/01/web-linking/
% it is not sufficient for
% data to simply contain URIs for it to be “linked”. There must be a
% specification of the format that identifies those URIs as links, and either
% defines the link semantics or how they can be determined. The link might be
% part of a generic link construct like the Atom and HTML <link> elements,
% referencing a relation from the link relation registry that provides the link
% semantics. Alternatively, the link semantics might be defined in the data
% format, as was the case in the “next” property from our example.

% REST has four architectural constraints:
% separation of resource from representation,
% manipulation of resources by representations,
% self-descriptive messages, and
% hypermedia as the engine of application state.

% http://amundsen.com/hypermedia/hfactor/

% Hypermedia as the engine of application state
% http://www.infoq.com/articles/mark-baker-hypermedia

\begin{quotation}
  The model application is therefore an engine that moves from one state to the next by examining and choosing from among the alternative state transitions in the current set of representations.
\end{quotation}\cite[sec. 5.3, p.103]{Fielding2000}

\subsection{Hypermedia in OpenSocial}

Webfinger, e.g. get a profile picture from an email address

Danger: One can trigger na http request by sending an email.

\section{Selection of components}

Apache Shindig for Open Social, includes client tests

http://code.google.com/p/kolab-android/

https://evolvis.org/projects/kolab-ws/

http://packages.ubuntu.com/source/maverick/dovecot-metadata-plugin
https://launchpad.net/ubuntu/+source/dovecot-metadata-plugin/8-0ubuntu1

% Apache Felix, Jackrabbit, RESTeasy http://blog.tfd.co.uk/2011/11/25/minimalist/
% Scala Dispatch HTTP requests http://dispatch.databinder.net/Dispatch.html
% Scala JSON serialization https://github.com/debasishg/sjson
% ATOM http://abdera.apache.org/ http://www.ibm.com/developerworks/xml/library/x-atompp3/ http://www.ibm.com/developerworks/xml/library/x-tipatom4/index.html

% JSON: http://jackson.codehaus.org/ http://code.google.com/p/google-gson/
% http://microformats.org/wiki/org.microformats.hCard

\subsection{REST framework}
Jersey recommended by \cite{Kaiser2011} above Restfulie and RESTeasy because of maturity and flexibility.

% http://www.torsten-horn.de/techdocs/jee-rest.htm RESTful Web Services mit JAX-RS und Jersey

Jersey has a atompub-contact client/server example app.

\subsection{VCard}

% http://sourceforge.net/projects/vcard4j is dead since 5
% years. http://sourceforge.net/projects/mime-dir-j forked and updated and is
% now also abandoned.
% http://sourceforge.net/projects/jpim/ dead since 2 years.
% active:
% http://code.google.com/p/android-vcard 
% http://sourceforge.net/projects/cardme/
% http://wiki.modularity.net.au/ical4j/index.php?title=VCard (easily extendable to XML, JSON)



\section{Testing}
How to test the ReST/CardDAV interface?

% http://code.google.com/p/rest-client/

Portable Contacts test client at plaxo \url{http://www.plaxo.com/pdata/testClient}

\url{http://code.google.com/p/rest-assured/} \url{http://restfuse.com/}

\appendix

\section{Standards}
\subsection{Contacts / Persons}

% http://schema.org/Person

% http://www.ibiblio.org/hhalpin/homepage/notes/vcardtable.html
\begin{description}[\breaklabel\setleftmargin{1ex}]

  \item[RFC 6450 vCard Format Specification]
    This document defines the vCard data format for representing and exchanging
    a variety of information about individuals and other entities (e.g.,
    formatted and structured name and delivery addresses, email address,
    multiple telephone numbers, photograph, logo, audio clips, etc.). This is
    the new version and obsoletes RFCs 2425, 2426, and 4770, and updates RFC
    2739.

  \item[RFC 6351 xCard: vCard XML Representation]
    This document defines the XML schema of the vCard data format. 

  % http://portablecontacts.net/draft-spec.html
  % http://docs.opensocial.org/display/OSD/Specs
  % http://docs.opensocial.org/display/OSD/Enterprise+OpenSocial+Extensions link to calendar!
  % Mozilla erwägt PoCo http://groups.google.com/group/mozilla.dev.webapi/browse_thread/thread/3bd36f73336ce783?pli=1
  % https://code.google.com/apis/contacts/docs/poco/1.0/developers_guide.html
  \item[Portable Contacts, OpenSocial] 
    Portable Contacts defines contact data structures and a ReST API. It has
    been integrated in the OpenSocial standard.

  % http://www.nuxeo.com/en/resource-center/Videos/Nuxeo-World-2011/Leveraging-Open-Social-within-the-Nuxeo-Platform
  % http://wiki.magnolia-cms.com/display/WIKI/Magnolia+OpenSocial+Container
  % http://www.zdnet.com/blog/hinchcliffe/opensocial-20-will-key-new-additions-make-it-a-prime-time-player-in-social-apps/1603
  % http://www.cmswire.com/cms/social-business/open-standards-the-future-of-opensocial-20-013335.php
  % http://docs.opensocial.org/display/OSD/List+of+OpenSocial+Containers
  % http://www.informationweek.com/thebrainyard/news/industry_analysis/232200026
  % http://www.atlassian.com/opensocial/

  \item[Nepomuk Semantic Desktop Contact Ontology]

  % http://xmlns.com/foaf/spec/
  \item[Friend of a friend (FOAF)] 
    FOAF is a 

  % http://microformats.org/wiki/hcard
  \item[hCard]

  % http://microformats.org/wiki/jcard
  \item[jCard]

\end{description}

\subsection{Calendaring}
%\subparagraph{IETF (RFC)}
\begin{description}[\breaklabel\setleftmargin{1ex}]

  \item[RFC 5545 Internet Calendaring and Scheduling Core Object Specification]

    iCalendar is the core data schema for calendaring information. This is the
    new version and obsoletes RFC2445

  \item[RFC 6321 xCal: The XML format for iCalendar]

    This specification defines a format for representing iCalendar data in
    XML. More specifically, is to define an XML format that allows iCalendar
    data to be converted to XML, and then back to iCalendar, without losing any
    semantic meaning in the data. Anyone creating XML calendar data according to
    this specification will know that their data can be converted to a valid
    iCalendar representation as well.

  \item[CalWS RESTful Web Services Protocol for Calendaring]

    This document, developed by the XML Technical Committee, specifies a RESTful
    web services Protocol for calendaring operations. This protocol has been
    contributed to OASIS WS-CALENDAR as a component of the WS-CALENDAR
    Specification under development by OASIS.

  % https://code.google.com/apis/calendar/v3
  \item[Google Calendar API V3]

    While not being a standard, the Google Calendar API is RESTful and will
    surely be implemented by many client applications. It's remarkable that the
    API supports partial GETs returning only specified fields and the HTTP PATCH
    verb to update only specified fields.

  % http://open-services.net/specifications/
  \item[Open Services for Lifecycle Collaboration (OSLC)]

    uses FOAF person \url{http://open-services.net/bin/view/Main/OSLCCoreSpecAppendixA?sortcol=table;up=#foaf_Person_Resource}

    provides change management, some overlapping to iCal TODOs \url{http://open-services.net/bin/view/Main/CmSpecificationV2}

    reference implementation: \url{http://eclipse.org/lyo}

\end{description}

\subsection{Scheduling}

\begin{description}[\breaklabel\setleftmargin{1ex}]
  \item[RFC 5546 iCalendar Transport-Independent Interoperability Protocol (iTIP)] 

    Scheduling Events, BusyTime, To-dos and Journal Entries; Specifies
    the mechanisms for calendaring event interchange between calendar
    servers. This is the new version and obsoletes RFC2446

  \item[RFC 6047 iCalendar Message-Based Interoperability Protocol (iMIP)]

    Specifies how to exchange calendaring data via e-mail. This is the new
    version and obsoletes RFC2447.

\end{description}

\subsection{Relations and Links}
% http://code.google.com/apis/socialgraph/
\begin{description}[\breaklabel\setleftmargin{1ex}]

  % http://gmpg.org/xfn/
  \item[Xhtml Friends Network (XFN)] 

    One of the relations returned by Google's webfinger.

  % https://datatracker.ietf.org/doc/draft-jones-appsawg-webfinger/
  \item[Webfinger]
    Webfinger in Firefox Contacts Add-On \url{http://mozillalabs.com/blog/2010/03/contacts-in-the-browser-0-2-released/}

  \item[RFC 6415 Web Host Metadata]

  % http://docs.oasis-open.org/xri/xrd/v1.0/xrd-1.0.html
  % http://en.wikipedia.org/wiki/XRDS
  % http://code.google.com/p/webfinger/wiki/CommonLinkRelations
  % http://hueniverse.com/category/discovery/
  \item[Extensible Resource Descriptor (XRD)] 

\end{description}

\subsection{out of scope}
\begin{description}[\breaklabel\setleftmargin{1ex}]

  % LDIF for person info

  % http://www.hr-xml.org
  % http://de.wikipedia.org/wiki/HR-XML  
  \item[HR XML]

    The HR-XML Consortium is the only independent, non-profit, volunteer-led
    organization dedicated to the development and promotion of a standard suite
    of XML specifications to enable e-business and the automation of human
    resources-related data exchanges.

  % http://www.openmobilealliance.org/Technical/release_program/cab_v1_0.aspx
  \item[OMA Converged Address Book V1.0]

    Standard by the Open Mobile Alliance defining data structures and
    synchronization of contact data. It references vCard.
  
  % http://en.wikipedia.org/wiki/Open_Collaboration_Services
  \item[Open Collaboration Services]

    Also contains data structures for persons and events but does not reuse any
    known standard. See this thread:
    \url{http://lists.freedesktop.org/archives/ocs/2011-December/000136.html}

  % http://www.w3.org/TR/contacts-api
  \item[W3C Contacts API]

    A standard on how address books cold be accessed on devices or from
    JavaScript inside a Web Browser. The standard references vCard, OMA
    Converged Address Book and Portable Contacts.

  % http://www.w3.org/TR/vcard-rdf/
  \item[W3C vCard ontology]

  % http://www.w3.org/2000/10/swap/pim/contact
  \item[W3C PIM ontology]

\end{description}


\section{People, Groups and Organizations}
% http://lists.w3.org/Archives/Public/public-device-apis/ - Contacts API
% 
% https://www.ietf.org/mailman/listinfo/calsify
% https://www.ietf.org/mailman/listinfo/ischedule - only 8 mails since 2009
% https://www.ietf.org/mailman/listinfo/httpmail only 3 mails since 2009
% https://www.ietf.org/mailman/listinfo/vcarddav
% https://www.ietf.org/mailman/listinfo/caldav
% https://www.ietf.org/mailman/listinfo/imap5

%http://groups.google.com/group/portablecontacts

%http://tech.groups.yahoo.com/group/rest-discuss

\paragraph{People}
\begin{description}[\breaklabel\setleftmargin{1ex}]

  \item[Eran Hammer-Lahav]
      \url{http://hueniverse.com}
      Yahoo!, OAuth

  \item[Eliot Lear <lear@cisco.com>]
      IETF Calsify WG chair

  \item[Julian Reschke <julian.reschke@gmx.de>]
% Julian Reschke, WebDAV Experte, RFC 5995, greenbytes GmbH,Hafenweg 16, 48155 Münster , Germany

  \item[Lisa Dusseault]
      
    Lisa Dusseault is a development manager and standards architect at the Open
    Source Applications Foundation, where she's involved in the Chandler, Cosmo
    and Scooby projects. Previously, Lisa came from Xythos, an Internet startup
    where she was development manager for four years. She has also been an IETF
    contributor on various Internet applications protocols for eight years now,
    and continues to do this kind of work at OSAF. She co-chairs the IETF IMAP
    extensions and CALSIFY (Calendaring and Scheduling Standards Simplification)
    Working Groups. She is also the author of a book on WebDAV and co-author of
    CalDAV, an open and interoperable protocol for calendar access and sharing.

  \item[Mark Nottingham]
%  http://www.mnot.net/personal/

  \item[Mike Amundsen <mamund@yahoo.com>]
    \url{http://amundsen.com}

  \item[Mike Conley]

    \url{http://mikeconley.ca/blog/}
    % Email: mike.d.conley@gmail.com
    % Twitter: http://www.twitter.com/mike_conley
    % IRC: You can usually find me on Freenode as m_conley
    working on a new address book for Thunderbird: \url{https://wiki.mozilla.org/Thunderbird/tb-planning}

  \item[Peter Saint-Andre <stpeter@stpeter.im>]

    IETF Calsify WG area director

  \item[Joseph Smarr]

    former Plaxo now Google
    presentation about portable contacts at vcarddav wg http://tools.ietf.org/agenda/74/slides/vcarddav-2.pdf
    http://josephsmarr.com
    http://anyasq.com/79-im-a-technical-lead-on-the-google+-team

% http://notizblog.org/2011/11/17/the-long-term-failure-of-openweb/




\end{description}



\section{Implementations}

% http://wiki.portablecontacts.net/w/page/17776143/Software%20and%20Services%20using%20Portable%20Contacts
% http://docs.opensocial.org/display/OSD/List+of+OpenSocial+Containers

% http://en.wikipedia.org/wiki/List_of_applications_with_iCalendar_support
% http://syncevolution.org/
% http://www.janrain.com/solutions/supported-networks
% http://code.google.com/p/caldav4j/
% http://www.webdav.org/projects/
% http://en.wikipedia.org/wiki/CardDAV
% webdav server http://milton.ettrema.com
% http://jackrabbit.apache.org/jackrabbit-webdav-library.html
% http://davmail.sourceforge.net/ Exchange GateWay using Jackrabbit
% http://en.wikipedia.org/wiki/List_of_applications_with_iCalendar_support
% Open Core: http://en.wikipedia.org/wiki/Open_core
% http://en.wikipedia.org/wiki/Groupware
\subsection{Servers}
\begin{description}[\breaklabel\setleftmargin{1ex}]

  % http://en.wikipedia.org/wiki/Cyn.in
  \item[Cyn.in]
    Python, Open Core

  % http://www.davical.org/
  \item[DAViCal] 

    PHP, SQL storage, CalDAV, CardDav

  \item[eGroupWare]

  % http://en.wikipedia.org/wiki/EXo_Platform
  \item[eXo Platform]
    Open Core, Java, AGPL, participates in OpenSocial?

  % http://en.wikipedia.org/wiki/Group-Office
  \item[Group-Office]
    PHP, AGPL

  \item[Horde]

  % obm.org http://en.wikipedia.org/wiki/OBM_Groupware
  \item[OBM Groupware]
    PHP, GPL

  \item[Open-Xchange]
    Java, 
    In 2006 a Debian packaging attempt was canceled because upstream decided not to publish security updates for the open source version anymore.\footnote{\citeurl{http://web.archive.org/web/20100510133805/http://seraphyn.deveth.org/archives/10-Keine-Zukunft-in-der-freien-Version-von-Open-Exchange-auf-Debian.html}{2011-12-19}}

  % http://owncloud.org
  \item[owncloud]

    ownCloud supports syncing of calendar and contacts information via the
    CalDAV and CardDAV protocols.

  % http://en.wikipedia.org/wiki/Scalix
  \item[Scalix]
    Open Core
    Scalix Public License (SPL) based on MPL, requires to show the Scalix Logo

  % http://en.wikipedia.org/wiki/Simple_Groupware
  \item[Simple Groupware]
    PHP, GPL, SQL

  % http://en.wikipedia.org/wiki/SOGo
  \item[SOGo]
    CalDAV and CardDAV, written in Objective-C

  % http://en.wikipedia.org/wiki/Tiki_Wiki_CMS_Groupware
  \item[Tiki Wiki]
    PHP, SQL
    Contacts \url{http://doc.tiki.org/Contacts}, Calendar \url{http://doc.tiki.org/Calendar}
    iCal export
    apparently no CardDAV/CalDAV
    many many features!

  % http://en.wikipedia.org/wiki/Tine_2.0
  \item[Tine 2.0]
    Tine is not eGroupWare

  % http://en.wikipedia.org/wiki/Zarafa_%28software%29
  \item[Zarafa]
     PHP, MySQL
     IIRC it uses an Entity-Attribute-Value pattern to store its data in the relational db.

  % http://en.wikipedia.org/wiki/Zimbra
  \item[Zimbra]
    Open Core, Own license (Zimbra Public License),
    RFP since 2008 open: http://bugs.debian.org/cgi-bin/bugreport.cgi?bug=498316
    

\end{description}

\subsection{Clients}

\begin{description}[\breaklabel\setleftmargin{1ex}]

  % http://en.wikipedia.org/wiki/Spicebird
  \item[Spicebird]
    built on top of Thunderbird with Calendar

  \item[Thunderbird]

    CardDAV via SoCO connector \url{http://www.sogo.nu/fr/downloads/frontends.html}

  \item[WebiCal]
   % http://code.google.com/p/webical/
     Java, YUI, Web frontend for a CalDAV server, uses iCal4J

  \item[Evolution, Evolution Data Server]
  \item[KDE Kontact, Akonadi]

  \item[more CardDAV] \url{http://wiki.davical.org/w/CardDAV/Clients} \url{http://en.wikipedia.org/wiki/CardDAV#Implementations}
  \item[more CalDAV]  \url{http://wiki.davical.org/w/CalDAV_Clients} \url{http://en.wikipedia.org/wiki/CalDAV#Implementations}

\end{description}


\subsection{Web Services}
% Google Calendar http://code.google.com/apis/calendar/caldav/

\subsection{Others}

\section{Links}

\begin{itemize}
\item \url{http://thesocialweb.tv}
\item \url{http://www.vogella.de/articles/REST/article.html} REST with Java (JAX-RS) using Jersey - Tutorial
\item \url{https://addons.mozilla.org/de/firefox/addon/restclient/}
\item \url{http://dataportability.org/} still active?
\item \url{http://tech.groups.yahoo.com/group/rest-discuss/messages/17242?threaded=1&m=e&var=1&tidx=1} REST and Semantic
% http://exist.sourceforge.net/
% http://wiki.davical.org/w/CardDAV/Configuration/Well-known_URLs
% https://github.com/karl/monket-google-calendar A simplified UI for Google Calendar.
% Nuxeo switches from Python to Java: http://www.infoq.com/articles/nuxeo_python_to_java http://www.infoq.com/news/nuxeo-zope-java-migration
% JAXB Tutorial http://docs.oracle.com/cd/E17802_01/webservices/webservices/docs/1.6/tutorial/doc/JAXBWorks2.html
% XML Schema http://www.javaworld.com/javaworld/jw-08-2005/jw-0808-xml.html?page=2
% https://github.com/jaliss/securesocial provides OAuth, OAuth2 and OpenID authentication for Play Framework
% Oauth http://code.google.com/intl/de/apis/accounts/docs/OAuth2.html
% Permissions compared. IMAP, WEBDAV, ... http://chandlerproject.org/bin/view/Journal/LisaDusseault20040409

\end{itemize}

IANA link relations registry \url{http://www.iana.org/assignments/link-relations/link-relations.xml}

ATOM landscape overview \url{http://dret.typepad.com/dretblog/atom-landscape.html}

\subsection{Apache Shindig}
RPC vs. REST API for Shindig/OpenSocial: \url{http://groups.google.com/group/opensocial-and-gadgets-spec/browse_thread/thread/a4ddf7cd09f90237/5cfa1658e1c1d698?lnk=gst&q=rest#5cfa1658e1c1d698}, \url{http://groups.google.com/group/opensocial-and-gadgets-spec/browse_thread/thread/d1a5627fb6e686ce/d27d47dee92a87b2} One argument was support for batching. A restful batching proposal didn't get consensus: \url{https://docs.google.com/View?docid=dc43mmng_23fdbpp7hd&pli=1}

Flow of REST requests in Shindig \url{https://sites.google.com/site/opensocialarticles/Home/shindig-rest-java}

Google+ is likely to become OpenSocial enabled: \url{http://groups.google.com/group/opensocial-and-gadgets-spec/browse_thread/thread/1187241df6759a9a}

Shindig issues to implement OpenSocial 2.0 \url{https://docs.google.com/spreadsheet/ccc?key=0AihdZBncP3KzdGN3dVl3MFpIUlk2TXIyR3hfUDhHZUE&hl=en_US#gid=0}

How Shindig supports extensions: \url{https://cwiki.apache.org/confluence/display/SHINDIG/Arbitrary+Extensions+to+Apache+Shindig%27s+Data+Model}

Videos about some 2.0 OS features \url{http://groups.google.com/group/opensocial-and-gadgets-spec/browse_thread/thread/7b911edfb1bb3b4d}

OS and RDF \url{http://groups.google.com/group/opensocial-and-gadgets-spec/browse_thread/thread/20f62d627003509b}

OpenSocial Development Environment (OSDE, Eclipse Plugin)  \url{https://sites.google.com/site/opensocialdevenv}

\subsection{Socialsite}

Oracle's (former Sun's) extension to Apache Shindig. Blog \url{http://blogs.oracle.com/socialsite}

\section{TODO}
\begin{itemize}
\item Does funambol.org has interesting implementations?
\end{itemize}
% LDAP wäre auch eine Möglichkeit für Addressdaten, aber:
% kein Standardschema, Mozilla anders als Apple
% manche Clients können evtl. Adressen aus LDAP lesen aber nicht schreiben.

% Calendaring is not easy as can be seen by the impressive list of failed projects:
% http://www.hula-project.org/ 
% Dreaming in Code - Scott Rosenberg's software epic. about the chandler failure
% http://xmpp.org/extensions/xep-0054.html

% http://en.wikibooks.org/wiki/LaTeX/Glossary

support Plain Text Format (text/plain), RFC5147 URI fragment identifier for plain text?

\begin{quote}
  Sowohl Atom als auch AtomPub definieren XML-Vokabulare, die eine Erweiterung
  mit zwei Mechanismen unterstützen. Zum einen ist im Standard definiert, dass
  neue Elemente in diesen Vokaularen selbst von standardkonformen Prozessoren
  ignoeriert werden müssen. [...] Gleichzeitig ist es überall dort, wo es nicht
  explizit verboten ist, möglich, Elemente ausanderen XML-Namespaces
  einzubetten.
\end{quote}\cite[p. 102]{Tilkov2011}


\bibliography{references}{}
\bibliographystyle{alphadin}
\end{document}

% Local Variables:
% ispell-dictionary: "american"
% eval: (progn (flyspell-mode 1) (outline-minor-mode 1) (goto-address-mode 1) (hide-body))
% End:
%  LocalWords:  RESTful
