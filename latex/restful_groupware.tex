\documentclass[12pt,a4paper]{scrartcl}		% KOMA-Klassen benutzen!

%\usepackage[ngerman]{babel}			% deutsche Namen/Umlaute
\usepackage[utf8]{inputenc}			% Zeichensatzkodierung
\usepackage{url}
\usepackage{graphicx}
\usepackage[colorlinks=false, pdfborder={0 0 0}]{hyperref}
\usepackage{amsmath}
\usepackage{multicol}
\usepackage{glossaries}
\usepackage{expdlist}

\usepackage{setspace} % Anderthalbfacher Zeilenabstand ist Standard in den meisten Seminararbeiten. Das Paket setspace ermöglicht ein einfaches Umstellen von normalem, anderthalbfachen oder sogar doppeltem Zeilenabstand. 
\usepackage[paper=a4paper,inner=25mm,outer=20mm,top=15mm,bottom=20mm]{geometry} %Das geometry Paket dient zur Einrichtung der Seiten. Hier werden die jeweiligen Seitenränder angegeben. Diese wWerte sollten durch die jeweiligen Vorgaben des Seminarleiters oder Instituts ersetzt werden.
\setlength{\parindent}{1.7em} %Neue Abschnitte werden mit hängendem Einzug gesetzt, parindent definiert. um wie viel der Absat eingerückt wird. Die Einheit em ist abhängig vom verwendeten Zeichensatz und daher absoluten Werten in mm oder cm vorzuziehen. 
\setcounter{secnumdepth}{3} %Bis zu welcher Gliederungsebene nummeriert werden soll gibt dieser Befehl vor. In diesem Falle werden \section, \subsection und \subsubsection nummereiert.
\setcounter{tocdepth}{3} %Bis zu welcher Ebene Einträge ins Inhaltsverzeichnis aufgenommen werden. In diesem Beispiel ebenfalls bis Ebene drei (\subsubsection). Ein durch \paragraph ausgewiesener Abschnitt wird demnach nicht im Inhaltsverzeichnis auftauchen. 

\newcommand{\citeurl}[2]{\url{#1} (#2)}



\begin{document}
%\titlehead{}
%\subject{subject}
\title{}
\subtitle{}
\author{Thomas Koch\\\url{thomas@koch.ro}\\matriculation number 7250371}
\publishers{Fernuniversität Hagen\\Faculty of mathematics and computer science}
\date{\today}
%\thanks{}
\maketitle{}

%\newpage{}
\tableofcontents{}
\begin{abstract}

\end{abstract}
\newpage{}

\section{Motivation}

Why vCard/CardDav: many clients

Why OpenSocial / Portable Contacts:
\begin{itemize}
\item used by Google, LinkedIn,
\item used in Enterprise applications like Attlassian tools (Jira, Confluence, ...), Nuxeo CMS, ...
\item OpenSocial can be used to implement inhouse portals and populate it with data from the companies GroupWare
\end{itemize}

\section{Is webdav restful?}

Roy Fielding says no: \url{http://tech.groups.yahoo.com/group/rest-discuss/message/5874}

\begin{quotation}
PROP* methods conflict with REST because they prevent
important resources from having URIs and effectively double the
number of methods for no good reason. Both Henrik and I argued
against those methods at the time. It really doesn't matter
how uniform they are because they break other aspects of the
overall model, leading to further complications in versioning
(WebDAV versioning is hopelessly complicated), access control
(WebDAV ACLs are completely wrong for HTTP), and just about every
other extension to WebDAV that has been proposed.

[\ldots]

The problem with MOVE is that it is actually an operation on two
independent namespaces (the source collection and destination
collection). The user must have permission to remove from the
source collection and add to the destination collection, which
can be a bit of a problem if they are in different authentication
realms. COPY has a similar problem, but at least in that case
only one namespace is modified. I don't think either of them map
very well to HTTP.
\end{quotation}

The discussion also continued on the microformats mailing list
\url{http://microformats.org/discuss/mail/microformats-rest/2006-April/thread.html#217}.

see Amundsen2010 for a restful approach to properties.

Is ATOM an alternative to WebDAV?

\begin{quote}
  AtomPub is different from DAV in two key respects:
  \begin{itemize}
  \item The client doesn't control where things go, the server does
  \item It is allowed and expected that an AtomPub server will look at the incoming information and change it (generate ID, timestamps, sanitize HTML, etc)
  \end{itemize}
\end{quote}
Tim Bray, http://www.imc.org/atom-protocol/mail-archive/msg11271.html



Examples of APIs that use REST instead of WebDAV:
\begin{itemize}
\item Amazon S3 \url{http://awsmedia.s3.amazonaws.com/pdf/RESTandS3.pdf} \url{http://docs.amazonwebservices.com/AmazonS3/latest/dev/RESTAPI.html}

\end{itemize}


\section{Kolab and its use of IMAP}
Kolab uses an IMAP server as the data store and
synchronization protocol for calendar and contact informations. I want to
compare this approach to a restful one.

Advantages of IMAP:
\begin{itemize}
\item already there, since Mail uses it
\item can store blobs/files so no need to map the iCal/vCard files to a relational scheme
\item out of the box support for offline work and later synchronization (How does it solve editing conflicts?)
\end{itemize}
Disadvantages of IMAP:
\begin{itemize}
\item Complicate, 38 RFCs according to \url{http://de.wikipedia.org/wiki/Internet_Message_Access_Protocol} see also: \url{http://www.apps.ietf.org/rfc/ipoplist.html}
\item All clients directly access the iCal/vCard files with no moderation layer in between. This means that no validation or normalization can be done. Schema updates can only be done if all clients cooperate.
\item IMAP imposes a folder structure. Google's gmail is an example for another, tag based approach. Messages could have several tags. It is therefor hard to access Gmail via IMAP.
\item Sam Varshavchik, the author of the courier Mail Transfer Agent explains the history of IMAP and claims that the IMAP standard is broken: http://www.courier-mta.org/fud/
\item IMAP is so complicate that the IMAP wiki holds 10 pages of advises for IMAP client authors: http://www.imapwiki.org/ClientImplementation RFC 2683 ``IMAP4 Implementation Recommendations'' is a 23 pages document (cut 5 pages for verbosity) explaining how to implement another RFC standard. Is there any widely used standard that needs another RFC explaining how to implement it?
\item \url{http://en.wikipedia.org/wiki/Internet_Message_Access_Protocol#Disadvantages}
\item Some attempts to create a simpler alternative to IMAP:
  \begin{itemize}
  \item http://en.wikipedia.org/wiki/POP4
  \item \url{http://en.wikipedia.org/wiki/Simple_Mail_Access_Protocol} also here http://www.courier-mta.org/cone/smap1.html
  \item \url{http://en.wikipedia.org/wiki/Internet_Mail_2000}
  \item HTTP restful: http://tools.ietf.org/id/draft-dusseault-httpmail-00.txt mailing list: https://www.ietf.org/mailman/listinfo/httpmail
  \item BikINI is not IMAP http://bikini.caterva.org
  \item Outlook uses HTTP to communicate with Hotmail
  \item another rest mail proposal: http://www.prescod.net/rest/restmail/
  \end{itemize}
\item more rants: http://blog.gaborcselle.com/2010/02/how-to-replace-imap.html
\item IMAP issues found by the chandler project http://chandlerproject.org/bin/view/Jungle/IntrinsicIMAPIssues
\end{itemize}

\section{Persistency for Groupware Data}
Relational Databases vs. NoSQL databases vs. plain files

Relational databases are not practical for contacts, events or todos. Common patterns in systems that use relational DBs for that purpose:
\begin{itemize}
\item artificial limits of entries, e.g. only 3 email addresses per contact, because there are only three columns email1, email2 and email3.
\item Fields for custom data like custom1 to customX
\item EAV pattern: tables like: id, foreign\_id, type, value
\end{itemize}
\section{Synchronizing a large collection}

How to efficiently synchronize a large collection of contacts with the server without checking each contact for changes?

Portable Contacts has a filter ``updatedSince''.

How is synchronization done in CardDAV?

\section{Mediatypes}

% http://blog.programmableweb.com/2011/11/18/rest-api-design-putting-the-type-in-content-type/
% http://www.wrml.org Web Resource Modeling Language


\subsection{Updates with alternative Mediatypes}
How to handle updates, if the mediatypes are not isomorph?

How does Google handle PATCH in the calendar API?

\subsection{Mediatype conversion}

Which fields of portable contacts are derived from vCard: \url{http://wiki.portablecontacts.net/w/page/17776141/schema}

\section{Hypermedia in RESTful applications}

% http://restpatterns.org/Articles/The_Hypermedia_Scale

% http://linkednotbound.net/2010/12/01/web-linking/
% it is not sufficient for
% data to simply contain URIs for it to be “linked”. There must be a
% specification of the format that identifies those URIs as links, and either
% defines the link semantics or how they can be determined. The link might be
% part of a generic link construct like the Atom and HTML <link> elements,
% referencing a relation from the link relation registry that provides the link
% semantics. Alternatively, the link semantics might be defined in the data
% format, as was the case in the “next” property from our example.

% REST has four architectural constraints:
% separation of resource from representation,
% manipulation of resources by representations,
% self-descriptive messages, and
% hypermedia as the engine of application state.

% http://amundsen.com/hypermedia/hfactor/

% Hypermedia as the engine of application state
% http://www.infoq.com/articles/mark-baker-hypermedia

\begin{quotation}
  The model application is therefore an engine that moves from one state to the next by examining and choosing from among the alternative state transitions in the current set of representations.
\end{quotation}\cite[sec. 5.3, p.103]{Fielding2000}

\subsection{Hypermedia in OpenSocial}

Webfinger, e.g. get a profile picture from an email address

Danger: One can trigger na http request by sending an email.

\section{Selection of components}

Apache Shindig for Open Social, includes client tests

http://code.google.com/p/kolab-android/

https://evolvis.org/projects/kolab-ws/

http://packages.ubuntu.com/source/maverick/dovecot-metadata-plugin
https://launchpad.net/ubuntu/+source/dovecot-metadata-plugin/8-0ubuntu1

% Apache Felix, Jackrabbit, RESTeasy http://blog.tfd.co.uk/2011/11/25/minimalist/
% Scala Dispatch HTTP requests http://dispatch.databinder.net/Dispatch.html
% Scala JSON serialization https://github.com/debasishg/sjson

% JSON: http://jackson.codehaus.org/ http://code.google.com/p/google-gson/

\subsection{REST framework}
Jersey recommended by \cite{Kaiser2011} above Restfulie and RESTeasy because of maturity and flexibility.

% http://www.torsten-horn.de/techdocs/jee-rest.htm RESTful Web Services mit JAX-RS und Jersey

\section{Testing}
How to test the ReST/CardDAV interface?

% http://code.google.com/p/rest-client/

Portable Contacts test client at plaxo \url{http://www.plaxo.com/pdata/testClient}

\appendix

\section{Standards}
\subsection{Contacts / Persons}

% http://schema.org/Person

% http://www.ibiblio.org/hhalpin/homepage/notes/vcardtable.html
\begin{description}[\breaklabel\setleftmargin{1ex}]

  \item[RFC 6450 vCard Format Specification]

    This document defines the vCard data format for representing and exchanging
    a variety of information about individuals and other entities (e.g.,
    formatted and structured name and delivery addresses, email address,
    multiple telephone numbers, photograph, logo, audio clips, etc.). This is
    the new version and obsoletes RFCs 2425, 2426, and 4770, and updates RFC
    2739.

  \item[RFC 6351 xCard: vCard XML Representation]

    This document defines the XML schema of the vCard data format. 

  % http://portablecontacts.net/draft-spec.html
  % http://docs.opensocial.org/display/OSD/Specs
  % http://docs.opensocial.org/display/OSD/Enterprise+OpenSocial+Extensions link to calendar!
  % Mozilla erwägt PoCo http://groups.google.com/group/mozilla.dev.webapi/browse_thread/thread/3bd36f73336ce783?pli=1
  % https://code.google.com/apis/contacts/docs/poco/1.0/developers_guide.html
  \item[Portable Contacts, OpenSocial] 

    Portable Contacts defines contact data structures and a ReST API. It has
    been integrated in the OpenSocial standard.

  % http://www.nuxeo.com/en/resource-center/Videos/Nuxeo-World-2011/Leveraging-Open-Social-within-the-Nuxeo-Platform
  % http://wiki.magnolia-cms.com/display/WIKI/Magnolia+OpenSocial+Container
  % http://www.zdnet.com/blog/hinchcliffe/opensocial-20-will-key-new-additions-make-it-a-prime-time-player-in-social-apps/1603
  % http://www.cmswire.com/cms/social-business/open-standards-the-future-of-opensocial-20-013335.php
  % http://docs.opensocial.org/display/OSD/List+of+OpenSocial+Containers
  % http://www.informationweek.com/thebrainyard/news/industry_analysis/232200026
  % http://www.atlassian.com/opensocial/

  \item[Nepomuk Semantic Desktop Contact Ontology]

  % http://xmlns.com/foaf/spec/
  \item[Friend of a friend (FOAF)] 

    FOAF is a 

  % http://microformats.org/wiki/hcard
  \item[hCard]

\end{description}

\subsection{Calendaring}
%\subparagraph{IETF (RFC)}
\begin{description}[\breaklabel\setleftmargin{1ex}]

  \item[RFC 5545 Internet Calendaring and Scheduling Core Object Specification]

    iCalendar is the core data schema for calendaring information. This is the
    new version and obsoletes RFC2445

  \item[RFC 6321 xCal: The XML format for iCalendar]

    This specification defines a format for representing iCalendar data in
    XML. More specifically, is to define an XML format that allows iCalendar
    data to be converted to XML, and then back to iCalendar, without losing any
    semantic meaning in the data. Anyone creating XML calendar data according to
    this specification will know that their data can be converted to a valid
    iCalendar representation as well.

  \item[CalWS RESTful Web Services Protocol for Calendaring]

    This document, developed by the XML Technical Committee, specifies a RESTful
    web services Protocol for calendaring operations. This protocol has been
    contributed to OASIS WS-CALENDAR as a component of the WS-CALENDAR
    Specification under development by OASIS.

  % https://code.google.com/apis/calendar/v3
  \item[Google Calendar API V3]

    While not being a standard, the Google Calendar API is RESTful and will
    surely be implemented by many client applications. It's remarkable that the
    API supports partial GETs returning only specified fields and the HTTP PATCH
    verb to update only specified fields.

  % http://open-services.net/specifications/
  \item[Open Services for Lifecycle Collaboration (OSLC)]

    uses FOAF person \url{http://open-services.net/bin/view/Main/OSLCCoreSpecAppendixA?sortcol=table;up=#foaf_Person_Resource}

    provides change management, some overlapping to iCal TODOs \url{http://open-services.net/bin/view/Main/CmSpecificationV2}

    reference implementation: \url{http://eclipse.org/lyo}

\end{description}

\subsection{Scheduling}

\begin{description}[\breaklabel\setleftmargin{1ex}]
  \item[RFC 5546 iCalendar Transport-Independent Interoperability Protocol (iTIP)] 

    Scheduling Events, BusyTime, To-dos and Journal Entries; Specifies
    the mechanisms for calendaring event interchange between calendar
    servers. This is the new version and obsoletes RFC2446

  \item[RFC 6047 iCalendar Message-Based Interoperability Protocol (iMIP)]

    Specifies how to exchange calendaring data via e-mail. This is the new
    version and obsoletes RFC2447.

\end{description}

\subsection{Relations and Links}
% http://code.google.com/apis/socialgraph/
\begin{description}[\breaklabel\setleftmargin{1ex}]

  % http://gmpg.org/xfn/
  \item[Xhtml Friends Network (XFN)] 

    One of the relations returned by Google's webfinger.

  % https://datatracker.ietf.org/doc/draft-jones-appsawg-webfinger/
  \item[Webfinger]
    Webfinger in Firefox Contacts Add-On \url{http://mozillalabs.com/blog/2010/03/contacts-in-the-browser-0-2-released/}

  \item[RFC 6415 Web Host Metadata]

  % http://docs.oasis-open.org/xri/xrd/v1.0/xrd-1.0.html
  % http://en.wikipedia.org/wiki/XRDS
  % http://code.google.com/p/webfinger/wiki/CommonLinkRelations
  % http://hueniverse.com/category/discovery/
  \item[Extensible Resource Descriptor (XRD)] 

\end{description}

\subsection{out of scope}
\begin{description}[\breaklabel\setleftmargin{1ex}]

  % http://www.openmobilealliance.org/Technical/release_program/cab_v1_0.aspx
  \item[OMA Converged Address Book V1.0]

    Standard by the Open Mobile Alliance defining data structures and
    synchronization of contact data. It references vCard.
  
  % http://www.w3.org/TR/contacts-api
  \item[W3C Contacts API]

    A standard on how address books cold be accessed on devices or from
    JavaScript inside a Web Browser. The standard references vCard, OMA
    Converged Address Book and Portable Contacts.

  % http://www.w3.org/TR/vcard-rdf/
  \item[W3C vCard ontology]

  % http://www.w3.org/2000/10/swap/pim/contact
  \item[W3C PIM ontology]

  % http://www.hr-xml.org
  % http://de.wikipedia.org/wiki/HR-XML  
  \item[HR XML]

    The HR-XML Consortium is the only independent, non-profit, volunteer-led
    organization dedicated to the development and promotion of a standard suite
    of XML specifications to enable e-business and the automation of human
    resources-related data exchanges.

\end{description}


\section{People, Groups and Organizations}
% http://lists.w3.org/Archives/Public/public-device-apis/ - Contacts API
% 
% https://www.ietf.org/mailman/listinfo/calsify
% https://www.ietf.org/mailman/listinfo/ischedule - only 8 mails since 2009
% https://www.ietf.org/mailman/listinfo/httpmail only 3 mails since 2009
% https://www.ietf.org/mailman/listinfo/vcarddav
% https://www.ietf.org/mailman/listinfo/caldav
% https://www.ietf.org/mailman/listinfo/imap5

%http://groups.google.com/group/portablecontacts

%http://tech.groups.yahoo.com/group/rest-discuss

\paragraph{People}
\begin{description}[\breaklabel\setleftmargin{1ex}]

  \item[Eliot Lear <lear@cisco.com>]
      IETF Calsify WG chair

  \item[Lisa Dusseault]
      
    Lisa Dusseault is a development manager and standards architect at the Open
    Source Applications Foundation, where she's involved in the Chandler, Cosmo
    and Scooby projects. Previously, Lisa came from Xythos, an Internet startup
    where she was development manager for four years. She has also been an IETF
    contributor on various Internet applications protocols for eight years now,
    and continues to do this kind of work at OSAF. She co-chairs the IETF IMAP
    extensions and CALSIFY (Calendaring and Scheduling Standards Simplification)
    Working Groups. She is also the author of a book on WebDAV and co-author of
    CalDAV, an open and interoperable protocol for calendar access and sharing.

  \item[Peter Saint-Andre <stpeter@stpeter.im>]

    IETF Calsify WG area director

  \item[Joseph Smarr]

    former Plaxo now Google
    presentation about portable contacts at vcarddav wg http://tools.ietf.org/agenda/74/slides/vcarddav-2.pdf
    http://josephsmarr.com
    http://anyasq.com/79-im-a-technical-lead-on-the-google+-team

  \item[Mike Conley]

    \url{http://mikeconley.ca/blog/}
    % Email: mike.d.conley@gmail.com
    % Twitter: http://www.twitter.com/mike_conley
    % IRC: You can usually find me on Freenode as m_conley
    working on a new address book for Thunderbird: \url{https://wiki.mozilla.org/Thunderbird/tb-planning}


% http://notizblog.org/2011/11/17/the-long-term-failure-of-openweb/

% Julian Reschke, WebDAV Experte, RFC 5995, greenbytes GmbH,Hafenweg 16, 48155 Münster , Germany
% Mark Nottingham http://www.mnot.net/personal/

\end{description}



\section{Implementations}

% http://wiki.portablecontacts.net/w/page/17776143/Software%20and%20Services%20using%20Portable%20Contacts
% http://docs.opensocial.org/display/OSD/List+of+OpenSocial+Containers

% http://en.wikipedia.org/wiki/List_of_applications_with_iCalendar_support
% http://syncevolution.org/
% http://www.janrain.com/solutions/supported-networks
% http://vcard4j.sourceforge.net/
% http://code.google.com/p/caldav4j/
% http://www.webdav.org/projects/
% http://en.wikipedia.org/wiki/CardDAV
% webdav server http://milton.ettrema.com
% http://jackrabbit.apache.org/jackrabbit-webdav-library.html
% http://davmail.sourceforge.net/ Exchange GateWay using Jackrabbit
% http://en.wikipedia.org/wiki/List_of_applications_with_iCalendar_support
% Open Core: http://en.wikipedia.org/wiki/Open_core
% http://en.wikipedia.org/wiki/Groupware
\subsection{Servers}
\begin{description}[\breaklabel\setleftmargin{1ex}]

  % http://en.wikipedia.org/wiki/Cyn.in
  \item[Cyn.in]
    Python, Open Core

  % http://www.davical.org/
  \item[DAViCal] 

    PHP, SQL storage, CalDAV, CardDav

  \item[eGroupWare]

  % http://en.wikipedia.org/wiki/EXo_Platform
  \item[eXo Platform]
    Open Core, Java, AGPL, participates in OpenSocial?

  % http://en.wikipedia.org/wiki/Group-Office
  \item[Group-Office]
    PHP, AGPL

  \item[Horde]

  % obm.org http://en.wikipedia.org/wiki/OBM_Groupware
  \item[OBM Groupware]
    PHP, GPL

  % http://owncloud.org
  \item[owncloud]

    ownCloud supports syncing of calendar and contacts information via the
    CalDAV and CardDAV protocols.

  % http://en.wikipedia.org/wiki/Scalix
  \item[Scalix]
    Open Core
    Scalix Public License (SPL) based on MPL, requires to show the Scalix Logo

  % http://en.wikipedia.org/wiki/Simple_Groupware
  \item[Simple Groupware]
    PHP, GPL, SQL

  % http://en.wikipedia.org/wiki/SOGo
  \item[SOGo]
    CalDAV and CardDAV, written in Objective-C

  % http://en.wikipedia.org/wiki/Tiki_Wiki_CMS_Groupware
  \item[Tiki Wiki]
    PHP, SQL
    Contacts \url{http://doc.tiki.org/Contacts}, Calendar \url{http://doc.tiki.org/Calendar}
    iCal export
    apparently no CardDAV/CalDAV
    many many features!

  % http://en.wikipedia.org/wiki/Tine_2.0
  \item[Tine 2.0]
    Tine is not eGroupWare

  % http://en.wikipedia.org/wiki/Zarafa_%28software%29
  \item[Zarafa]

  % http://en.wikipedia.org/wiki/Zimbra
  \item[Zimbra]
    Open Core, Own license (Zimbra Public License),
    RFP since 2008 open: http://bugs.debian.org/cgi-bin/bugreport.cgi?bug=498316
    

\end{description}

\subsection{Clients}

\begin{description}[\breaklabel\setleftmargin{1ex}]

  % http://en.wikipedia.org/wiki/Spicebird
  \item[Spicebird]
    built on top of Thunderbird with Calendar

  \item[Thunderbird]

    CardDAV via SoCO connector \url{http://www.sogo.nu/fr/downloads/frontends.html}

  \item[Evolution, Evolution Data Server]
  \item[KDE Kontact, Akonadi]

  \item[more CardDAV] \url{http://wiki.davical.org/w/CardDAV/Clients} \url{http://en.wikipedia.org/wiki/CardDAV#Implementations}
  \item[more CalDAV]  \url{http://wiki.davical.org/w/CalDAV_Clients} \url{http://en.wikipedia.org/wiki/CalDAV#Implementations}

\end{description}


\subsection{Web Services}
% Google Calendar http://code.google.com/apis/calendar/caldav/

\subsection{Portable Contacts}

\section{Links}

\begin{itemize}
\item \url{http://thesocialweb.tv}
\item \url{http://www.vogella.de/articles/REST/article.html} REST with Java (JAX-RS) using Jersey - Tutorial
\item \url{https://addons.mozilla.org/de/firefox/addon/restclient/}
% http://exist.sourceforge.net/
% http://wiki.davical.org/w/CardDAV/Configuration/Well-known_URLs
% https://github.com/karl/monket-google-calendar A simplified UI for Google Calendar.
% Nuxeo switches from Python to Java: http://www.infoq.com/articles/nuxeo_python_to_java http://www.infoq.com/news/nuxeo-zope-java-migration
% JAXB Tutorial http://docs.oracle.com/cd/E17802_01/webservices/webservices/docs/1.6/tutorial/doc/JAXBWorks2.html
% XML Schema http://www.javaworld.com/javaworld/jw-08-2005/jw-0808-xml.html?page=2
% https://github.com/jaliss/securesocial provides OAuth, OAuth2 and OpenID authentication for Play Framework
% Oauth http://code.google.com/intl/de/apis/accounts/docs/OAuth2.html
% Permissions compared. IMAP, WEBDAV, ... http://chandlerproject.org/bin/view/Journal/LisaDusseault20040409

\end{itemize}

\subsection{Apache Shindig}
RPC vs. REST API for Shindig/OpenSocial: \url{http://groups.google.com/group/opensocial-and-gadgets-spec/browse_thread/thread/a4ddf7cd09f90237/5cfa1658e1c1d698?lnk=gst&q=rest#5cfa1658e1c1d698}, \url{http://groups.google.com/group/opensocial-and-gadgets-spec/browse_thread/thread/d1a5627fb6e686ce/d27d47dee92a87b2} One argument was support for batching. A restful batching proposal didn't get consensus: \url{https://docs.google.com/View?docid=dc43mmng_23fdbpp7hd&pli=1}

Flow of REST requests in Shindig \url{https://sites.google.com/site/opensocialarticles/Home/shindig-rest-java}

Fielding says OpenSocial is not Restful \url{http://roy.gbiv.com/untangled/2008/rest-apis-must-be-hypertext-driven}. Discussion: \url{http://groups.google.com/group/opensocial-and-gadgets-spec/browse_thread/thread/aff4ba7373e21284/201a413efa67c26e#201a413efa67c26e}

Google+ is likely to become OpenSocial enabled: \url{http://groups.google.com/group/opensocial-and-gadgets-spec/browse_thread/thread/1187241df6759a9a}

Shindig issues to implement OpenSocial 2.0 \url{https://docs.google.com/spreadsheet/ccc?key=0AihdZBncP3KzdGN3dVl3MFpIUlk2TXIyR3hfUDhHZUE&hl=en_US#gid=0}

How Shindig supports extensions: \url{https://cwiki.apache.org/confluence/display/SHINDIG/Arbitrary+Extensions+to+Apache+Shindig%27s+Data+Model}

Videos about some 2.0 OS features \url{http://groups.google.com/group/opensocial-and-gadgets-spec/browse_thread/thread/7b911edfb1bb3b4d}

OS and RDF \url{http://groups.google.com/group/opensocial-and-gadgets-spec/browse_thread/thread/20f62d627003509b}

\subsection{Socialsite}

Oracle's (former Sun's) extension to Apache Shindig. Blog \url{http://blogs.oracle.com/socialsite}

\section{TODO}
\begin{itemize}
\item Does funambol.org has interesting implementations?
\end{itemize}
% LDAP wäre auch eine Möglichkeit für Addressdaten, aber:
% kein Standardschema, Mozilla anders als Apple
% manche Clients können evtl. Adressen aus LDAP lesen aber nicht schreiben.

% Calendaring is not easy as can be seen by the impressive list of failed projects:
% http://www.hula-project.org/ 
% Dreaming in Code - Scott Rosenberg's software epic. about the chandler failure
% http://xmpp.org/extensions/xep-0054.html

% http://en.wikibooks.org/wiki/LaTeX/Glossary
\bibliography{references}{}
\bibliographystyle{alphadin}
\end{document}

% Local Variables:
% ispell-dictionary: "american"
% eval: (progn (flyspell-mode 1) (outline-minor-mode 1) (goto-address-mode 1)  (hide-body))
% End:
%  LocalWords:  RESTful
