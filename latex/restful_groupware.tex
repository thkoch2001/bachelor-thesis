\documentclass[12pt,a4paper]{scrartcl}		% KOMA-Klassen benutzen!

%\usepackage[ngerman]{babel}			% deutsche Namen/Umlaute
\usepackage[utf8]{inputenc}			% Zeichensatzkodierung
\usepackage{url}
\usepackage{graphicx}
\usepackage{amsmath}
\usepackage{multicol}
\usepackage{glossaries}
\usepackage{expdlist}
\usepackage{subfig}
\usepackage{listings}
\usepackage{tikz}

\usepackage{setspace} % Anderthalbfacher Zeilenabstand ist Standard in den meisten Seminararbeiten. Das Paket setspace ermöglicht ein einfaches Umstellen von normalem, anderthalbfachen oder sogar doppeltem Zeilenabstand. 
\usepackage[paper=a4paper,inner=25mm,outer=20mm,top=15mm,bottom=20mm]{geometry} %Das geometry Paket dient zur Einrichtung der Seiten. Hier werden die jeweiligen Seitenränder angegeben. Diese wWerte sollten durch die jeweiligen Vorgaben des Seminarleiters oder Instituts ersetzt werden.
\setlength{\parindent}{1.7em} %Neue Abschnitte werden mit hängendem Einzug gesetzt, parindent definiert. um wie viel der Absat eingerückt wird. Die Einheit em ist abhängig vom verwendeten Zeichensatz und daher absoluten Werten in mm oder cm vorzuziehen. 
\setcounter{secnumdepth}{3} %Bis zu welcher Gliederungsebene nummeriert werden soll gibt dieser Befehl vor. In diesem Falle werden \section, \subsection und \subsubsection nummereiert.
\setcounter{tocdepth}{3} %Bis zu welcher Ebene Einträge ins Inhaltsverzeichnis aufgenommen werden. In diesem Beispiel ebenfalls bis Ebene drei (\subsubsection). Ein durch \paragraph ausgewiesener Abschnitt wird demnach nicht im Inhaltsverzeichnis auftauchen. 

\usepackage[colorlinks=false, pdfborder={0 0 0}, plainpages=false]{hyperref}

\newcommand{\citeurl}[2]{\url{#1} (#2)}

\begin{document}
%\titlehead{}
%\subject{subject}
\title{}
\subtitle{}
\author{Thomas Koch\\\url{thomas@koch.ro}\\matriculation number 7250371}
\publishers{Fernuniversität Hagen\\Faculty of mathematics and computer science}
\date{\today}
%\thanks{}
\maketitle{}

%\newpage{}
\tableofcontents{}
\begin{abstract}

\end{abstract}
\newpage{}


\section{Motivation}

Although computers became ubiquitous for some time now, they still don't help
their users with their most basic information management needs: Make contacts,
calendars, notices and to do items available across different devices and share
them with my peers.

Existing solutions are either based on non-free software (Microsoft Outlook),
brittle and unreliable\footnote{See comments on the individual projects in
  appendix ...} or require the user to trust his personal data to the commercial
interests of a multinational corporation.\footnote{Experiences with Android and Data in the cloud \citeurl{http://keithp.com/blogs/calypso}{?}}

This work uses an ongoing effort to draft a meta model for restful
applications. Applying this meta model to the domain of this work may provide
further insights into its general applicability and fit.

\section{Requirements}

  \begin{itemize}
  \item Lesen/Schreiben der verwalteten Resourcen: Kontakte, Kalender-Events, Todos, Journal-Einträge, Free-Busy, \ldots
  \item Synchronisation von Collections für Offline-Nutzung
  \item einfach zu implementieren (vgl. CalDAV, CardDAV, IMAP) $\rightarrow$ ReST
  \item Standardkonform $\rightarrow$ xCard, xCal, ATOM
  \item nutzbar durch JavaScript: JSON basierte Medientypen
  \item Groupware Elemente: Kontakte, Kalender-Events, Todos, Journal-Einträge, Free-Busy
  \end{itemize}

% restful: one entry point, reuse of existing protocols and media types
% easy to understand and implement
% Collections z.B. für Projekte könnten alle möglichen Typen konsumieren.
These requirements include the features of CardDAV.\cite[sec. 1]{RFC6352}

% SRS template from http://code.google.com/p/e-bibliophile/wiki/SRS
\subsection{Scope}
% This subsection should
% a)Identify the software product(s) to be produced by name (e.g., Host DBMS, 
% Report Generator, etc.);
% b)Explain what the software product(s) will, and, if necessary, will not do;
% c)Describe the application of the software being specified, including relevant 
% benefits, objectives, and goals;
% d)Be consistent with similar statements in higher level specifications 
% (e.g., the system requirements specification), if they exist.

This work defines a protocol to share 

\subsection{Definitions}

Kolab is the name of a software product ... TODO

A couple of related terms and concepts exist that all more or less overlap with
the functionality provided by Kolab: Groupware, Personal Information
Management/Manager (PIM), Group Information Management (GIM), Computer-supported
cooperative/collaborative work, Knowledge management, (Enterprise) Content
Management.

TODO: keinen der Terme benutzen.

\subsection{General Requirements}
% This subsection of the SRS should provide a summary of the major functions 
% that the software will perform. For example, an SRS for an accounting program 
% may use this part to address customer account maintenance, customer statement, 
% and invoice preparation without mentioning the vast amount of detail that each 
% of those functions requires.
% Sometimes the function summary that is necessary for this part can be taken 
% directly from the section of the higher level specification (if one exists) 
% that allocates particular functions to the software product. Note that for the 
% sake of clarity
% a)The functions should be organized in a way that makes the list of functions 
% understandable to the customer or to anyone else reading the document for the 
% first time.
% b)Textual or graphical methods can be used to show the different functions 
% and their relationships. Such a diagram is not intended to show a design of 
% a product, but simply shows the logical relationships among variables.

\paragraph{Restful}
The application should obey the constraints of a rest application as specified
in \cite{Fielding2000}.

TODO: 4 Grundconstraints von REST auflisten.

The above constraints are not an end in itself but lead to the following
required or desirable properties:

\begin{itemize}
\item Cacheability (5.1.4) can keep the data available also in offline mode, improves performance and scalability.
\item Simplicity helps to develop glue code to connect the application to other systems or to extend it.
\item Modifiability allows to adapt the Groupware to changes in the organization.
\item Reliability should not need additional justification.
\item Administrative scalability means that intermediary components can be deployed independent of the administrator of the main application.
\end{itemize}

Other outcomes described by Fielding that may not be of importance for the
present work are: scalability in terms of users, network performance and
efficiency.



\subsection{User Classes and Characteristics}

TODO: plain Web Browsers, Desktop applications (PIM Suites), JavaScript Gadgets/Widgets (see OpenSocial) probably with WebStorage\footnote{\citeurl{http://www.w3.org/TR/webstorage/}{2012-2-2}}

Aus den verschiedenen Benutzerern leitet sich ab, dass verschiedene Medientypen unterstützt werden sollen: xCard, JSON, HTML

\subsection{Operation Environment}

The application is expected to be installed in a Java servlet container like Tomcat or Jetty and to contact a separate storage component. The primarily targeted storage component is an IMAP server with a Kolab conform set of groupware folders. However the design should not restrict the extension to a document database like CouchDB, plain files, relational or XML databases.

\subsection{Design and Implementation Constraints}

\subsection{Specific Requirements}



\subsection{Excluded requirements}

\paragraph{Search}
It is not required that the server implements any means to search its data. It
is not excluded that such a facility could be added later. It is however
expected that searching could be implemented separately. This could be done
either on a synchronizing client or as a separate system in the same
administrative domain as the server.

\paragraph{Performance optimization}
The system is meant to inherit the benefits of a restful architecture. It should
therefor be possible to attach separate caching intermediaries for read
requests. Rather then concentrating on the performance of the implementation of
read requests it should be taken care that the architecture supports external
caching and thus avoids to serve the same read request multiple times.

\paragraph{Access Control}
The aspect of access control would broaden the scope of this work to
wide. However it could be kept in mind, whether the proposed design could be
enhanced by a separate access control design as proposed in
\cite{conf/rest/GrafZLW11}.

\paragraph{Versioning}
WebDAV and therefor CalDAV and CardDAV support the versioning of resources as an
extension to the HTTP protocol. Versioning is an important feature for a text
authoring system that may have been the main target for the WebDAV protocol.  It
does however seem to be of little use for the resources considered here. The
resources are mostly created in one session by one user and seldom modified.

\paragraph{Locking}
As with Versioning, this is feature of WebDAV is not considered. Instead of
locking a resource HTTP supports conditional updates and leaves conflict
resolution to the client.

\paragraph{Push notifications}
This work does not include any means to actively notify (push) a client about
changes happening on the server. The client needs to initiate a request (pull)
to the server to look for changes. However separate solutions exist\footnote{most
  notable PubSubHubBub
  \citeurl{http://code.google.com/p/pubsubhubbub/}{2012-1-5}} to enable a push
workflow on top of a feed based
application.\cite{Wilde:2009:FQP:1693155.1693220} It may therefor not be seen as a disadvantage to omit push notifications as a requirement.\footnote{\cite[sec. 1]{RFC6352} explicitly mentions missing ``change notifications'' as a ``key disadvantage'' of CardDAV.}


\section{Media Types}
% http://amundsen.com/hypermedia/
% http://martinfowler.com/articles/richardsonMaturityModel.html
% http://code.google.com/p/implementing-rest/wiki/RMM
% http://looselyconnected.wordpress.com/2011/03/09/the-richardson-maturity-model-of-rest-and-roy-fielding/


\begin{quote}
  To some extent, people get REST wrong because I failed to include enough
  detail on media type design within my dissertation.~--~Roy T. Fielding
\end{quote}
from: Rest APIs must be hypertext driven\footnote
  {\citeurl{http://roy.gbiv.com/untangled/2008/rest-apis-must-be-hypertext-driven}{2011-12-20}}

\cite[sec. 7.2]{Pautasso:2008:RWS:1367497.1367606} identifies the support of different media types as an issue that "can complicate and hinder the interoperability" and "requires more maintenance effort".

\cite{Davis:2011:XTR:1967428.1967437} proposes a XML based REST framework that uses XForms, XQuery, XProc, XSLT and an XML database. It can benefit from the constraint that it only supports XML based media types. It is to be seen, which ideas from this work could be reused in the case of a broader variety of supported media types.

\subsection{Syntax vs. Semantic}

The usage of standardized media types is one key difference between an API and a
restful API. \cite[sec. 5.2.1.2]{Fielding2000} Only if the client has knowledge
about the media type can it do something meaningful with it besides just
receiving it. In that sense, the often used mime types application/xml or
application/json are not really media types. They don't tell the browser or user
anything meaningful beside the \emph{syntax} of the data.\footnote{
\citeurl{http://blog.programmableweb.com/2011/11/18/rest-api-design-putting-the-type-in-content-type}{2011-21-20}
and Web Resource Modeling Language \citeurl{http://www.wrml.org}{2011-12-20} both by Mark Massé
}

To do anything meaningful with plain json or xml, the client programmer must
normally look up the meaning or \emph{semantic} of the data in the API
documentation. The data therefor fails the self-descriptive constraint of
REST.

Compare this with a mime type like \texttt{application/atom+xml}. It specifies
the syntax (xml) and the semantic (atom) of the data. Of course somebody once
needed to read the atom specification and program the client with the knowledge
of how to process this media type. The purpose of standardized media types
however is that there is a limited number of them and that they are reused by many
sites and clients.

Large sites like Google, Facebook or Twitter can successfully attract enough
developers to read their specifications and program clients accordingly. Usually
one can also find client libraries that can help a lot. REST however envisions a
decentralized web in which parties can interact without previous knowledge of
each other. This becomes possible through the usage of well known predefined
media types.

%  in the area of semantic
% standardization. Questions whether and how contact information should also hold
% space for the place of birth and place and date of death of a
% person\footnote{\citeurl{https://datatracker.ietf.org/doc/draft-ietf-vcarddav-birth-death-extensions/}{2011-12-20}}
% are independent from a data serialization format. So is the question whether and
% how to specify the sex or gender of a
% person.\footnote{\citeurl{http://www.ietf.org/mail-archive/web/vcarddav/current/msg01778.html}{2011-12-20}}


\subsection{Data Models of Media Types}

TODO:
\begin{itemize}
\item Ein generelles Daten Modell wäre hilfreich, um alle Medien Typen darauf zu projezieren und mit einer solchen Projektion dann innerhalb der Applikation zu arbeiten
\item Ein allgemeines Datenmodell könnte auch eine Hilfe sein als Zwischenschritt für Conversions zwischen Medientypen
\item Es gibt kein allgemeines, sinnvolles Datenmodell für alle Medientypen
\item Trotzdem können bestimmte hilfreiche Generalisierungen vorgenommen werden
  \begin{itemize}
  \item Die meisten Resourcen haben bestimmte generische Metadaten die entweder im Medientyp kodiert werden können oder mit dem Medientyp zusammen persistiert werden müssen
  \item Diese Metadaten finden sich auch in atom:entry wieder und sind: Autor, Updated, Titel, Summary, etag, id, name, links
  \item Transitional Links vs Structural Links: \url{http://java.net/projects/jax-rs-spec/pages/Hypermedia}
  \item Different categories of data: CSV, binary/plain text, large binary (video), tree (XML/JSON) (Referenz?)


  \end{itemize}
\end{itemize}

\subsubsection{XML vs. JSON}

This section investigates the two most common data models used by media types
and the issues that arise if an application needs to support both of them.

The application section of the IANA mime type registration has 294 entries
ending in ``+xml'' and only 3 ending in
``+json''.\footnote{\citeurl{http://www.iana.org/assignments/media-types/application/index.html}{2011-12-20}}
This stands in contrast to the rise of public JSON APIs and the decline of XML
APIs.\footnote{\citeurl{http://blog.programmableweb.com/2011/05/25/1-in-5-apis-say-bye-xml/}{2011-12-20} \citeurl{http://www.readwriteweb.com/cloud/2011/03/programmable-web-apis-popping.php}{2011-12-21}}

Web developers often prefer JSON over XML because it is perceived as easier to
parse, process and smaller in size. XML in comparison is seen as complicate,
slow to process and larger in size. A strong argument for JSON as the preferred
format for JavaScript applications is that JSON is a subset of
JavaScript.\footnote{ECMAScript for XML (E4X) makes XML a first class language
  construct in the browser but is only supported by Mozilla
  \citeurl{http://en.wikipedia.org/wiki/ECMAScript_for_XML}{2012-2-2}}

A drawback of this mismatch between the preference of media type designers and
API consumers is a possible duplication of work and incompatibilities across
different APIs. If two developers independently take the same XML media type as
a model to develop a JSON format, they can still come up with different
results. There is no standard mapping from XML schemes to JSON structures.

Instead, possible mappings have to trade of the preservation of all structural
information against the ``friendliness'' of the resulting JSON
structure.\cite{Boyer2011} Without going into detail, a JSON structure can be
seen as friendly if it makes best use of JSON's data types, is compact and easy
to process. \autoref{fig:waysmapxmljson} shows two different examples how to map
data from XML to JSON with one of them using JSON number values, being more
compact and probably easier to process.

\begin{figure}
\captionsetup[subfloat]{justification=raggedright,singlelinecheck=false}
\small{
\subfloat[XML fragment]{
  \begin{minipage}[b]{0.25\linewidth}
    \lstinputlisting{snippets/xml2json_xml}    
  \end{minipage}
}
\subfloat[unfriendly JSON]{
  \begin{minipage}[b]{0.4\linewidth}
    \lstinputlisting{snippets/xml2json_json1}
  \end{minipage}
}
\subfloat[less(?) unfriendly JSON]{
  \begin{minipage}[b]{0.25\linewidth}
    \lstinputlisting{snippets/xml2json_json2}
  \end{minipage}
}
}
  \caption{different ways to map XML to JSON}
  \label{fig:waysmapxmljson}
\end{figure}

Activity Streams has avoided the misalignment of an official XML format and an unofficial JSON deviate by defining an XML (ATOM) and JSON format from the beginning.\footnote{\citeurl{http://activitystrea.ms/}{2012-01-21}}

\subsection{vCard, iCalendar, xCard and xCal}

TODO:
\begin{itemize}
\item Textbasierte vs. XML Formate
\end{itemize}

Version 3 of vCard was published in 1998\cite{RFC2425} only a few months after
the W3C published Version 1.0 of XML\cite{Paoli:98:XR} and eight years before
JSON became an official standard.\cite{RFC4627}

\subsubsection{Hypermedia Support}

\begin{figure}[h]
  \usetikzlibrary{positioning,arrows,fit}
  \usetikzlibrary{decorations.pathmorphing,backgrounds,fit,decorations.pathreplacing}
  \begin{tikzpicture}
    [doc/.style={rectangle,draw=blue!30,fill=blue!20, node distance=6em}]
    \node[doc] (svc) [] {ServiceDoc};
    \node[doc] (collect) [below=of svc,text width=5em] {Contacts Collection}
      edge [<-] node {} (svc)
      edge [->, loop left] node {next} (collect);

    \node[doc] (calcollect) [below right=of svc,text width=5em] {Calendar Collection}
      edge [<-] node {} (svc);

    \node[doc] (entry) [below=of collect] {Entry (xCard)}
      edge [<-] node {1..n} (collect)
      edge [->] node [text width=5em] {personal calendar} (calcollect)
      edge [->, loop left] node {related xCards} (entry);

    \node[doc] (event) [below=of calcollect] {Entry (xCal event)}
      edge [<-] node {1..n} (calcollect)
      edge [->] node [below right] {participants} (entry);


  \end{tikzpicture}
  \caption{Hypermedia support in xCard and xCal}
\end{figure}


\subsection{Derived JSON formats for PIM data}

TODO Übereinstimmungen und Unterschiede vCard, portable contacts: Which fields of portable contacts are derived from vCard: \url{http://wiki.portablecontacts.net/w/page/17776141/schema}

Calendar Formate nur kurz

\subsection{Media Types for Collections}

Vergleich ATOM mit Medientyp Collection+JSON

JSON formats for collections:
Collection+JSON Mime-Type (approved in July 2011) by Mike Amundsen\footnote{\citeurl{http://amundsen.com/media-types/collection/}{2012-1-7}}
JSON ATOM serialization implemented by Apache Abdera\footnote{\citeurl{http://www.ibm.com/developerworks/library/x-atom2json/index.html}{2012-1-7} \citeurl{https://cwiki.apache.org/ABDERA/json-serialization.html}{2012-1-7}}


Some problems in loss-less conversion of ATOM to json:\cite{Snell2008}
\begin{itemize}
  \item JSON has no equivalent for the xml:lang attribute.
  \item Dereferencable IRIs must be transformed to URIs.
  \item URIs relative to an xml:base attribute must be resolved, also inside XHTML content elements.
  \item Repeatable elements must be converted to arrays.
  \item The ATOM date format (RFC 3339) differs from the JavaScript Date serialization.
  \item ATOM content elements are versatile but should be represented more meaningful in JSON then just a plain String.
  \item ATOM supports arbitrary extensions via namespaces.
\end{itemize}

\subsection{Media Type conversion}

Konvertieren zwischen verschiedenen Representationen die nach aussen gegeben werden vs. Konvertieren zwischen äußerer Representation und Representation für die Datenbank.

An welchen Punkten in der Architektur muss/kann/soll konvertiert werden?

\subsection{Updates with non isomorphic Media Types}
How to handle updates, if the mediatypes are not isomorph?

How does Google handle PATCH in the calendar API?

Schnittmenge der Medientypen identifizieren

Einen kanonischen Medientypen festlegen für Updates?

Erweiterungsmöglichkeiten v. Medientypen nutzen um Isomorphie herzustellen


\section{Design}

% The design must not support any possible use case but only restful web applications.

% The life cycle of ResourceHandlers is for the application. They receive the called URL as part of the request context as a method parameter.
% ResourceHandlers could be wired together at run time, instantiated with a set of handler functors which could carry their storage in their closure.
% Specifying the media types handled by a handler function as annotations is verbose. If I have a method that handles contacts, I need to specify several media types. If a method handler can handle Contacts, Events and Todos, I need to repeat a lot of media types.

% Handler functors return a tuppel of status code, headers set and response body.
% Alternatively they can return early by throwing a special exception carrying a status code and a message.
% Exceptions thrown from a ResourceHandler should be serialized in a way easily to consume for the client.

% Concerns: Authorization, Logging, Trigger Message (to indexing system, subscribers), Measure execution time, Support Cross-Origin Resource Sharing (CORS)\footnote{\citeurl{http://www.w3.org/TR/cors/}{2012-2-2}}, Compression

% The life cycle of a CollectionStorage is across individual requests to facilitate caching or database connection reuse.

\subsection{Overview}

\subsubsection{Structural and Behavioral Rest Model}

TODO Modelierung der Anwendung mit dem Meta Model nach Schreier.

Primary Resources: Contacts, Calendars, ...
List Resources: 

\subsubsection{AtomPub for PIM data}

Rob Yates \url{mailto:robert_yates@us.ibm.com}:
CalATOM \citeurl{http://robubu.com/?p=6}{2012-01-05} Draft:
CardATOM \citeurl{http://robubu.com/?p=10}{2012-01-05}

Rob Yates drafted a CalATOM
specification.\cite{draft-yates-atompub-calatom-00.txt} It references another
old draft (draft-snell-atompub-feature-12) to mark a collection as a CalATOM
collection. This seems unnecessary since we can specify that a collection
accepts calendar resources and thus marking a collection as a calendar.

Furthermore the draft mentions a capability to query for events in a given time
range. There is no equivalent for such a request for contacts.

In rest the draft only repeats or references the atompub standard and the xcal
format. This draft could therefor be seen as indication that the existing
standards are almost sufficient to define GroupWare APIs.

Google Data API also uses ATOM, but puts their tags directly inside the ``atom:entry'' tag 
instead of putting all of the content in the content element of an atom
entry.\footnote{\citeurl{http://web.archive.org/web/20081120001246/http://www.snellspace.com/wp/?p=314}{2012-01-05}}

\subsection{Components}

\paragraph{Dispatcher}

The dispatcher selects the Java method (see \ref{sec:components-actions}) that
should handle the request. The selection can depend at least on the path
component of the requested URI, the media types accepted by the client as
indicated in the request's ACCEPT header and the HTTP verb.

Every project implementing JAX-RS\cite{JAX-RS1.1} needs to have some kind of
dispatcher component. The specification itself does not identify this
component. It does however specify the algorithm a dispatcher needs to follow
and a set of Java annotations which must be used to configure the
dispatch. These annotations (PATH, GET for the HTTP verb and Produces) are
demonstrated in listing \ref{fig:jaxrs-annotated-resource-example}.

\begin{lstlisting}[label=fig:jaxrs-annotated-resource-example,
                   language=java,
                   caption={Example of a JAX-RS annotated Resource class (by Marek Potociar)}]
@Path("atm/{cardId}")	
public class AtmResource {	
`  
  @GET 	
  @Path("balance")	
  @Produces("text/plain")	
  public String balance(@PathParam("cardId") String card,	
                        @QueryParam("pin") String pin) {	
    return Double.toString(getBalance(card, pin));	
  }
\end{lstlisting}
 
Alternative approaches to configure the dispatcher are not designated by
JAX-RS. One possible alternative would be to expose an API to manually add
dispatch routes at runtime and remove the corresponding annotations from the
source code. This could have several advantages\footnote{This section does not
  intend to dispute the advantages of annotation based configuration.}:

\begin{itemize}
\item The path under which a resource type is served is decoupled from the code
  defining the behavior of the resource. This could enable the reuse of resource
  classes or methods in other contexts.
\item The decision which media types can be consumed or produced may not depend
  solely on the resource class or method. A resource method may work on a domain
  specific data type and the set of supported media types may depend on the
  available converter between media types and the data type. A photo album
  resource may be able to consume any number of different image formats that
  another framework can convert to an internal image representation.
\item The list of supported media types could be created programmatically. This
  enables reuse of set of equivalent media types or combination of media type
  categories for example to combine the sets of image, video and audio media
  types.
\item The concept of resource classes could be replaced altogether. The life
  cycle of a resource class in JAX-RS defaults to the request scope. During one
  request only one resource method is called. Resource methods therefor by
  default don't share state through resource class attributes. It would therefor
  be possible to bind individual functors to dispatcher routes and thus
  composing the equivalent of a resource class at runtime.
\end{itemize}

\paragraph{Actions}
\label{sec:components-actions}

An action is basically the code that should be executed to respond to a client
request. An action receives all information about a request and is connected to
the application. It can use and manipulate the application state and produces a
data structure representing the response. It can be compared to the ``Request
method'' defined in JAX-RS.

\paragraph{ResourceConverters}
\label{sec:components-resourceconverters}

\paragraph{CollectionStorage}
\label{sec:collectionstorage}

The collection storage interface offers the necessary means to store and
retrieve resources. For clarity this interface is not further broken down into a
read-only part and a full read-write interface.

A collection storage is instantiated with the knowledge of the collection it is
responsible for. It therefor typically only returns resources that were
previously stored through it although it may share its underlying persistency
provider with other collection storage instances.

The life cycle of a collection storage is scoped to the application. It is
therefor possible to attach memory based caching to this component.

\paragraph{GenericResourceAttributes}
\label{sec:component-genericresourceattrib}




\subsection{Detailed Design Considerations}
\subsubsection{Inline feeds or feeds with links?}

ATOM entries can either carry the full content inline in the content tag or link to the content.

\begin{itemize}

\item feeds could be dynamically created and inline only those entries that the client has not yet seen as indicated by the provided ETag or timestamp.
\item Feeds with links will most probably result in many further requests by the client to get the individual entries
\item Inline feeds may cause unnecessary work and deliver entries that the client has already seen
\item A compromise could be to provide the first page of a paginated feed with links and all additional pages with inlined entries
\end{itemize}

A final decision about this subject should be done based on utilization and performance data of a real world installation.

\subsection{Client Design}

What needs a client to know, how does it need to work?

\subsection{Related work}
\subsubsection{IMAP used by Kolab}

Kolab uses an IMAP server as the data store and
synchronization protocol for calendar and contact informations. I want to
compare this approach to a restful one.

Advantages of IMAP:
\begin{itemize}
\item already there, since Mail uses it
\item can store blobs/files so no need to map the iCal/vCard files to a relational scheme
\item out of the box support for offline work and later synchronization (How does it solve editing conflicts?)
\end{itemize}
Disadvantages of IMAP:
\begin{itemize}
\item Complicate, 38 RFCs according to \url{http://de.wikipedia.org/wiki/Internet_Message_Access_Protocol} see also: \url{http://www.apps.ietf.org/rfc/ipoplist.html}
\item All clients directly access the iCal/vCard files with no moderation layer in between. This means that no validation or normalization can be done. Schema updates can only be done if all clients cooperate.
\item IMAP imposes a folder structure. Google's gmail is an example for another, tag based approach. Messages could have several tags. It is therefor hard to access Gmail via IMAP.
\item Sam Varshavchik, the author of the courier Mail Transfer Agent explains the history of IMAP and claims that the IMAP standard is broken: http://www.courier-mta.org/fud/
\item IMAP is so complicate that the IMAP wiki holds 10 pages of advises for IMAP client authors: http://www.imapwiki.org/ClientImplementation RFC 2683 ``IMAP4 Implementation Recommendations'' is a 23 pages document (cut 5 pages for verbosity) explaining how to implement another RFC standard. Is there any widely used standard that needs another RFC explaining how to implement it?
\item \url{http://en.wikipedia.org/wiki/Internet_Message_Access_Protocol#Disadvantages}
\item Some attempts to create a simpler alternative to IMAP:
  \begin{itemize}
  \item http://en.wikipedia.org/wiki/POP4
  \item \url{http://en.wikipedia.org/wiki/Simple_Mail_Access_Protocol} also here http://www.courier-mta.org/cone/smap1.html
  \item \url{http://en.wikipedia.org/wiki/Internet_Mail_2000}
  \item HTTP restful: http://tools.ietf.org/id/draft-dusseault-httpmail-00.txt mailing list: https://www.ietf.org/mailman/listinfo/httpmail
  \item BikINI is not IMAP http://bikini.caterva.org
  \item Outlook uses HTTP to communicate with Hotmail
  \item another rest mail proposal: http://www.prescod.net/rest/restmail/
  \end{itemize}
\item more rants: http://blog.gaborcselle.com/2010/02/how-to-replace-imap.html
\item IMAP issues found by the chandler project http://chandlerproject.org/bin/view/Jungle/IntrinsicIMAPIssues
\end{itemize}



\subsubsection{WebDAV, CalDAV, CardDAV}

\begin{quote}
Were I to propose CalDAV today it would probably be CalAtom  
\end{quote}
-- Lisa Dusseault, February 29, 2008\footnote{\citeurl{http://nih.blogspot.com/2008/02/nearly-two-years-ago-i-made-prediction.html}{2012-1-6}}

\begin{quote}
  I gave a big sigh of relief when I read that, and I hope that the CardDAV
  folks take this to heart. Some parts of WebDAV (e.g., properties [\ldots])
  deserve to be taken out back and shot -- although, as Lisa says, they were
  necessary because of the state of the art at the time. That doesn't mean we
  can't do better now.
\end{quote}
-- Ted Leung, March 6, 2008\footnote{\citeurl{http://www.sauria.com/blog/2008/03/06/google-contacts-and-carddav/}{2012-1-6}}


Roy Fielding says WebDAV is not restful: \url{http://tech.groups.yahoo.com/group/rest-discuss/message/5874}

\begin{quotation}
PROP* methods conflict with REST because they prevent
important resources from having URIs and effectively double the
number of methods for no good reason. Both Henrik and I argued
against those methods at the time. It really doesn't matter
how uniform they are because they break other aspects of the
overall model, leading to further complications in versioning
(WebDAV versioning is hopelessly complicated), access control
(WebDAV ACLs are completely wrong for HTTP), and just about every
other extension to WebDAV that has been proposed.

[\ldots]

The problem with MOVE is that it is actually an operation on two
independent namespaces (the source collection and destination
collection). The user must have permission to remove from the
source collection and add to the destination collection, which
can be a bit of a problem if they are in different authentication
realms. COPY has a similar problem, but at least in that case
only one namespace is modified. I don't think either of them map
very well to HTTP.
\end{quotation}

The discussion also continued on the microformats mailing list
\url{http://microformats.org/discuss/mail/microformats-rest/2006-April/thread.html#217}.

see \cite{Amundsen2010} for a restful approach to properties.

Is ATOM an alternative to WebDAV?

\begin{quote}
  AtomPub is different from DAV in two key respects:
  \begin{itemize}
  \item The client doesn't control where things go, the server does
  \item It is allowed and expected that an AtomPub server will look at the incoming information and change it (generate ID, timestamps, sanitize HTML, etc)
  \end{itemize}
\end{quote}
Tim Bray, http://www.imc.org/atom-protocol/mail-archive/msg11271.html

\subsubsection{OpenSocial}

Roy Fielding wrote a blog post about the ``SocialSite REST API'', stating that
it isn't restful at all but clearly an RPC style
API.\footnote{\citeurl{http://roy.gbiv.com/untangled/2008/rest-apis-must-be-hyperhtext-driven}{2011-12-06}}
Fielding was referring to SocialSite, which is however an implementation of the
OpenSocial specification. Dave Johnson, a contributor to SocialSite, reacted on
this critique by opening a discussion on an OpenSocial mailing
list:\footnote{\citeurl{http://groups.google.com/group/opensocial-and-gadgets-spec/browse_thread/thread/aff4ba7373e21284/201a413efa67c26e}{2011-12-06}}
\begin{quote}
  I must admit, it is not clear to me how OpenSocial REST API violates the six
  rules that Roy stated.
\end{quote}
The above quote warrants a short comment. I also thought before, that REST would
be so simple that there wouldn't be much need for further studying. Every web
developer has some understanding of URIs, HTTP and a bit less of Hypermedia. So
it is easy to fall into the trap that everything build on top of HTTP would be
restful. Now however, after some more reading about REST, I can easily find
violations of the REST constraints in the OpenSocial specification.

Restful APIs are modeled around resources, their representations and links
between them. The authors of the OpenSocial API however seem to have modeled
their API around the concept of services:\cite[Social API Server, sec 2, Services]{OSSpec2.0.1}
\begin{quote}
  OpenSocial defines several services for providing access to a container's data.
\end{quote}

\paragraph{Fielding's critique}
Fielding listed some rules that a restful API must obey, but did not give
explicit examples how OpenSocial violates this rules. The following section will
provide such examples.

\begin{quote}
  A REST API should not be dependent on any single communication protocol,
  [\ldots] any protocol element that uses a URI for identification must allow
  any URI scheme to be used for the sake of that identification.
  \textit{[Failure here implies that identification is not separated from
    interaction.]}
\end{quote}

OpenSocial defines a construct called ``REST-URI-Fragment'' which is a clear
violation of the above rule. This URI fragment is simply an encoding of
procedural parameters as elements of an HTTP URI:\cite[Core API Server, sec
2.1.1.2.2, REST-URI-Fragment]{OSSpec2.0.1}

\begin{quote}
  Each service type defines an associated partial URI format. The base URI for
  each service is found in the URI element associated with the service in the
  discovery document. Each service type accepts parameters via the URL
  path. Definitions are of the form:
  
  \verb:{a}/{b}/{c}:
\end{quote}

URIs can contain a query component that would be more appropriate to contain
parameters. This would also have made it clearer to see that the specification
actually defines services instead of resources. One test showing the misfit is
to ask how dot-segments ('.' and '..') inside the URI fragment are interpreted
and whether this conforms with the letter and spirit of the URI
standard.\cite[sec 3.3]{RFC3986} Another misfit can be seen in the URI fragment
to retrieve one or multiple albums. In this case the 'c' part in the quoted
definition above is actually a list of albums to retrieve separated by a slash.

Fielding's second bullet point most likely refers to the
\texttt{X-HTTP-Method-Override} header. This header is a widely
used\footnote{\citeurl{http://www.subbu.org/blog/2008/07/another-rest-anti-pattern}{2011-12-06}}
workaround to allow the use of other HTTP methods than GET and POST from HTML
forms or through firewalls.

The next two points again refer to a more serious issue:

\begin{quote}
  A REST API should spend almost all of its descriptive effort in defining the
  media type(s) used for representing resources and driving application
  state[\ldots].  \textit{[Failure here implies that out-of-band information is
    driving interaction instead of hypertext.]}  A REST API must not define
  fixed resource names or hierarchies[\ldots] \textit{[Failure here implies that
    clients are assuming a resource structure due to out-of band
    information[\ldots]].}
\end{quote}

The OpenSocial specification contains a lot of out-of-band information
describing how to form URIs to access information or which methods to use on
which URIs for different actions. This means that the OpenSocial API is not
simple or intuitive to use but requires a client developer to read a lot of
specification, thus violating the simplicity property of a restful
architecture. Since the URIs are fixed in the specification and necessarily also
in clients, the modifiability property is also violated.\cite[sec
2.3]{Fielding2000}

The following tables give some examples of the specified URIs:

\begin{table}[h]
\begin{tabular}{p{6.5cm} l p{10cm}}
  URI fragment & Method & Description \\
  %\hline
  \verb:/people/{User-Id}/@self: & GET & profile for User-Id \\
  \verb:/people/{User-Id}/@self: & DELETE & remove User \\
  \verb:/people/{User-Id}/{Group-Id}: & GET & full profiles of group members \\
  \verb:: & POST & Create relationship, target specified \newline by \verb:<entry><id>: in body \\
   & POST & Update Person \\
  \verb:/people/{Initial-User-Id}/: \newline \verb:{Group-Id}/{Related-User-Id}: & GET & ??? \\
  \verb:/people/@supportedFields: & GET & list of supported person profile fields \\
  \verb:/groups/{User-Id}[/{Group-Id}]: & GET & one or all groups of a user \\
   & PUT & update group \\
   & DELETE & delete group \\
  \verb:/groups/{User-Id}: & POST & create group \\
\end{tabular}
  \caption{URI fragments for peoples and groups in the OpenSocial REST API}
\end{table}

\begin{table}[h]
\begin{tabular}{p{9.5cm} l p{8cm}}
  URI fragment & Method & Description \\
  %\hline
  \verb:/albums/{User-Id}/@self: & POST & create album \\
  \verb:/albums/{User-Id}/{Group-Id}[/Album-Id]*: & GET & one or multiple albums \\
  \verb:/mediaItems/{User-Id}/{Group-Id}/{Album-Id}/: \newline \verb:{MediaItem-Id}: & GET & one mediaitem \\
  \verb:/mediaItem/{User-Id}/@self/{Album-Id}: (sic!) & POST & create mediaitem \\
\end{tabular}
  \caption{URI fragments for albums and mediaitems in the OpenSocial REST API}
  \label{tab:OSURIAlbums}
\end{table}

The last URI in \autoref{tab:OSURIAlbums} is obviously missing an ``s'' behind
\texttt{mediaItem}. This typo is present and unfixed in the OpenSocial spec
since Version 1.0, released in march 2010. This is of course not a big issue in
itself, but rather a sign that the specification is too verbose and does
over-specify things that should rather be auto-discovered through hyperlinks.

Fielding mentions in a comment to the same blog post that the OpenSocial API
``could be made so [restful] with some relatively small changes'' but does not
specify these changes. However some issues can easily be identified.

First the data structures defined in OpenSocial do not use URIs to refer to
other resources. Instead they use Object-Ids that must then be inserted in the
appropriate URI templates. Examples are the \texttt{recipients},
\texttt{senderId}, \texttt{collectionIds} of messages and the \texttt{ownerId}
of albums. The person structure does not contain fields referencing other
resources. Thus it does not obviously violate REST like the albums and
messages. However it does so even worse since there are hidden references only
defined out-of-band in the specification. One can retrieve the albums, relations
or messages of a user by filling in the \texttt{userId} in one of the specified
URI templates. If Users would just contain references to other resources related
to a user, the specification could already be shortened a lot.

Another missed opportunity for a much more intuitive API is the relation of
media items and albums. This seems to be poster child example for a collection
(album) to collection-element (media item) relation which could have made use of
the hierarchical character of URI paths. OpenSocial however requires the client
developer to use two different URI templates. (\autoref{tab:OSURIAlbums})

A not so small change to OpenSocial would be to either use already standardized
and registered media types where possible or to register new types where
necessary. It seems that there are some already existing media types that could
be a good fit for OpenSocial but only miss a canonical json representation for
easy consumption by javascript applications. These are vCard for
persons,\footnote{OpenSocial persons are based on portable contacts which in
  turn borrowed field names from vCards.} ATOM entries\cite{RFC4287} for
messages, activities and media items and ATOM categories, collections or
workspaces\cite{RFC5023} for albums and groups. It would probably be necessary
to add extensions to the mentioned media types but vCard and ATOM both already
anticipated this need and provided mechanisms to do so.

The use of the ATOM format could promote the adoption of OpenSocial because
developers could either reuse existing knowledge about ATOM or would be more
motivated to learn about a system that is based on an already widespread
format. In fact OpenSocial already mentions ATOM as a way to wrap OpenSocial
data. However this wrapping does not build extend and reuse ATOM semantics as
proposed above but just puts the OpenSocial data structures inside the
entry/content element of ATOM. This kind of misuse of ATOM does of course not
deliver any advantage on top of the existing plain JSON or XML
representations.\footnote{compare Bill de Hora, Extensions v Envelopes. 11/2009 
%\newline
  \citeurl{http://www.dehora.net/journal/2009/11/28/extensions-v-envelopes}{2011-21-07}}
Consequently the newest OpenSocial specification deprecates any reference to the
ATOM format.

In Algermissen's classification (~\ref{sec:algerm-class-http}), the OpenSocial
REST API would actually be ``HTTP-based Type I'' due to the lack of media types
and direct hyper links between related resources. Algermissen writes that this
level has the lowest possible initial cost of all HTTP APIs. Or in other words:
The OpenSocial specification authors did not have to invest a lot to come up
with this API specification but maintenance and evolution cost may be medium or
high.

\subsubsection{CAP}

RFC 4324 "Calendar Access Protocol (CAP)"

\begin{quotation}
  [\ldots] CAP failed because most of the active implementors at the 
time felt it had become too complex to implement - partly because it 
required implementing a "brand new" protocol (BEEP). The upshot of that was 
the CalDAV effort, which was based on an existing fairly well understood 
protocol - WebDAV. One of the benefits of using WebDAV was that it was easy 
to take "off the shelf" WebDAV software (server and client) and get a basic 
CalDAV implementation done very quickly. Indeed, several months after 
CalDAV was published, CalConnect held an interop event at which several 
server and client implementations were present and demonstrated basic 
interoperability - something that would have been hard to achieve with CAP.
\end{quotation}
-- Cyrus Daboo\footnote{\citeurl{http://lists.calconnect.org/pipermail/caldeveloper-l/2012-January/000135.html}{2012-01-04}}

\section{Reusable Components}

Reuse is of course in general a good thing. In the context of Model Driven
Development (MDD) and code generation it is especially import to identify code
that is general enough to be provided by a library of framework and does not
need to be generated.

Minimizing the generated code also minimizes the extend of drawbacks associated
with code generation, most importantly conflicts between updates by the code
generator and manual modification.



% We have a namespace at our disposal. HTTP suggests to interpret the path component of this namespace in a hierarchic way.
% The URI must be mapped to a resource, independent of the requested media type or HTTP verb. 
% It therefor makes sense to do the interpretation of the URI or the path independent from Media type or HTTP verb dispatch.

% wilde-fqas argues that feeds (collections) provide a general enough abstraction 

Concerns regarding Media Types that needs to be implemented differently for each different Media Type:
\begin{itemize}
\item validate the Media Type
\item provide accessors to read, write parts of the Media Type
\item serialize, deserialize the whole Media Type
\item converters to other formats
\item accessors to common interfaces (projection), e.g. common generic resource attributes or common attributes of a contact
\end{itemize}
% Attributes of a resource could be virtual or derived, e.g. the size of an image is derived from the binary image data.
% Images also contain additional data that could be exposed as attributes.



Candidate areas for re-usability:
\begin{itemize}
\item link building, URL parsing
\item HTML form building, parsing
\item generic properties of resources, id
\item resource types
\item question to storage: does resource still match ETag? Has changed since?
\item all links of a resource: Link: intern/extern/undefined, href, rel, title, text, media type
\item bool function matchesMediaType(), getMediaType() auf WrappedEntry
\item Prüfung, ob ein Update durchgeführt werden soll, gemäß ETAG, ifnotchanged
\item Möglichkeit, DatenKlassen mit DatenTypen zu definieren wie in eZ Publish um automatische Views und Edit Ansichten zu ermöglichen.
\item Creation of resources: POST to collection with SLUG Header, PUT to URI, normalization of SLUG Header
\item Pagination (building and parsing of next and previous URIs, implementation of RFC5005), querying the collections entries provider with the correct parameters (offset, limit).
\item Storage interface with transaction support. An application may for example need to notify an indexing component after some resource has been changed. -- No transaction support: Every action that must happen in a transaction together with the resource change must be handled by the storage layer, must be aware of the storage technologie.
\end{itemize}

% Im Gegensatz zu Bildern, wie in \cite{Schreier:2011:MRA:1967428.1967434} können Kontakte komplett inline sein.


\appendix

Why vCard/CardDav: many clients

Why OpenSocial / Portable Contacts:
\begin{itemize}
\item used by Google, LinkedIn,
\item used in Enterprise applications like Attlassian tools (Jira, Confluence, ...), Nuxeo CMS, ...
\item OpenSocial can be used to implement inhouse portals and populate it with data from the companies GroupWare
\end{itemize}

\section{Evaluation of APIs}

\subsubsection{Algermissen's Classification of HTTP-based APIs}
\label{sec:algerm-class-http}

Jan Algermissen, proposes a ``Classification of HTTP-based APIs'' in February 2010.\footnote{
\citeurl{http://nordsc.com/ext/classification_of_http_based_apis.html}{2011-12-08}} 
He identifies and names a five level order of HTTP based APIs. Each level adds
one constraint to be obeyed. The last level obeys all five constraints and is
RESTful.~(\autoref{tab:algermissenclassification})

\begin{table}[h]
  \begin{tabular}{p{0.15\textwidth} p{0.3\textwidth} p{0.4\textwidth}}
    level name & constraint violated & violation description \\
    \hline \\
    WS-* &   Identification of Resources & Only service endpoint is identified by URI. No resources exposed. \\
    RPC URI-Tunneling &    Manipulation of Resources through Representations & SOAP body contains operation name, message not transferred to manipulate resource state. \\
    HTTP-based Type I &    Self-Descriptive Messages & Message semantics depend on action specified in message body. \\
    HTTP-based Type II &    Hypermedia as the Engine of Application State & Application state machine known at design time. \\
  \end{tabular}
  \caption{Classification of HTTP-based APIs after Algermissen}
  \label{tab:algermissenclassification}
\end{table}

In addition to that, Algermissen also provides a list of ``Effect on System
Properties and Cost'' of the different API styles and acknowledges that the
initial costs of developing a REST API may be higher compared to the other
styles but lower in the long run.




\section{Persistency for Groupware Data}
Relational Databases vs. NoSQL databases vs. plain files

Relational databases are not practical for contacts, events or todos. Common patterns in systems that use relational DBs for that purpose:
\begin{itemize}
\item artificial limits of entries, e.g. only 3 email addresses per contact, because there are only three columns email1, email2 and email3.
\item Fields for custom data like custom1 to customX
\item EAV pattern: tables like: id, foreign\_id, type, value
\end{itemize}
\section{Synchronizing a large collection}

How to efficiently synchronize a large collection of contacts with the server without checking each contact for changes?

Portable Contacts has a filter ``updatedSince''.

How is synchronization done in CardDAV?

``Feed Paging for Efficient Feed Synchronization'' thread at ATOM-Syntax \url{http://www.imc.org/atom-syntax/mail-archive/msg20060.html}

HTTP Delata encoding with Feed: \citeurl{http://www.wyman.us/main/2004/09/using_rfc3229_w.html}{2012-1-6}, implementations: \citeurl{http://www.wyman.us/main/2004/09/implementations.html}{2011-1-6}

``There's also an advanced technique involving etags which keeps track of when
each etag was issued and if an old etag is sent with the request then
everything since then is sent back in the response. Someone will chime in
with the reference.'' Eric Scheid, \citeurl{http://www.imc.org/atom-syntax/mail-archive/msg19832.html}{2012-1-6}

Microsoft specification FeedSync (formerly Simple Sharing Extension) \url{http://feedsyncsamples.codeplex.com},
 comment \url{http://notes.kateva.org/2008/01/microsoft-feedsync-what-heck-is-it-and.html}
\url{http://web.archive.org/web/20081114142152/http://www.snellspace.com/wp/?p=818}


\section{Media Types}

\begin{table}
  \begin{tabular}{l c c c c c}
    type of data & XML  & JSON                      & semantic          & microformat & comment \\
    Calendar     & xCal & Google calendar API       & \url{http://www.w3.org/TR/rdfcal} & hCalendar & other: iCalendar  \\
    Contact      & xCard & portable contacts, jCard & friend of a friend & hCard & other: vCard \\
    Resume       & HR XML &                          & Description of a Career & hResume & \\    
  \end{tabular}
  \caption{data in different formats}
  \label{tab:data-formats}
\end{table}


% Open-Xchange provides an (unrestful) HTTP/JSON API which is used by its
% javascript
% frontend.\footnote{\citeurl{http://oxpedia.org/index.php?title=HTTP_API}{2011-19-12}}
% The comprehensive documentation does not indicate whether the data structures
% for tasks, appointments, reminders and contacts were inspired by any
% standards. In any case the API documentation is a good example of the need for
% standard mime types in JSON format. The use case for this API also shows
% similarities to the use of OpenSocial for intranet frontends.
\subsection{Media Type conversion}

RFC 5023  "Collections are represented as Atom Feeds"

Is conneg (content negotiation) useful?
No: Norman Walsh, 2003, it can lead to hard to debug bugs\citeurl{http://norman.walsh.name/2003/07/02/conneg}{2011-1-9},
    Joe Gregorio, 2003,I can't communicate the mime type to request to a third service if I can only give an URI\citeurl{http://bitworking.org/news/WebServicesAndContentNegotiation}{2011-1-9}
Yes: Jerome Louvel, 2006, I could additionally provide URIs that override the accept headers with query parameters like ?format=json.\citeurl{http://blog.noelios.com/2006/11/15/reconsidering-content-negotiation/}{2011-1-9}
    

\begin{quote}
  No single data representation is ideal for every client. This protocol defines representations for each resource in three widely supported formats, JSON [RFC4627], XML, and Atom [RFC4287] / AtomPub [RFC5023], using a set of generic mapping rules. The mapping rules allow a server to write to a single interface rather than implementing the protocol three times.
\end{quote}\cite[Core API Server]{OSSpec2.0.1}

% microformats to json converter \url{http://microformatique.com/optimus/}


In 2007, a project called microjson wanted to standardize json representations of microformat data structures.\footnote{\citeurl{http://notizblog.org/2007/09/16/microjson-microformats-in-json/}{2011-12-19}} 

The project identified the need for a json schema:\footnote{\citeurl{http://web.archive.org/web/20080524003749/http://microjson.org/wiki/Schemas}{2022-12-19}}
\begin{quote}
  If there are standard microJSON formats for transfer of certain datasets, there will be a need to validate that data to ensure that it is infact valid format. To validate a format you need something that details the structure, data content types and required data. Sounds like we'll be needing a schema for each microJSON format. 
\end{quote}


jCard example from microjson.org\footnote{\citeurl{http://web.archive.org/web/20080517003233/http://microjson.org/wiki/JCard}{2011-12-19}}
\begin{lstlisting}
{
"vcard":{
  "name":{
    "given":"John",
    "additional":"Paul",
    "family":"Smith"
  },
  "org":"Company Corp",
  "email":"john@companycorp.com",
  "address":{
    "street":"50 Main Street",
    "locality":"Cityville",
    "region":"Stateshire",
    "postalCode":"1234abc",
    "country":"Someplace"
  },
  "tel":"111-222-333",
  "aim":"johnsmith",
  "yim":"smithjohn"
}
\end{lstlisting}

\subsection{Example: vCard}

\begin{lstlisting}
   <?xml version="1.0" encoding="UTF-8"?>
   <vcards xmlns="urn:ietf:params:xml:ns:vcard-4.0">
     <vcard>
       <fn><text>Simon Perreault</text></fn>
       <n>
         <surname>Perreault</surname>
         <given>Simon</given>
         <additional/>
         <prefix/>
         <suffix>ing. jr</suffix>
         <suffix>M.Sc.</suffix>
       </n>
       <bday><date>--0203</date></bday>
       <anniversary>
         <date-time>20090808T1430-0500</date-time>
       </anniversary>
       <gender><sex>M</sex></gender>
       <lang>
         <parameters><pref><integer>1</integer></pref></parameters>
         <language-tag>fr</language-tag>
       </lang>
       <lang>
         <parameters><pref><integer>2</integer></pref></parameters>
         <language-tag>en</language-tag>
       </lang>
       <org>
         <parameters><type><text>work</text></type></parameters>
         <text>Viagenie</text>
       </org>
       <adr>
         <parameters>
           <type><text>work</text></type>
           <label><text>Simon Perreault
   2875 boul. Laurier, suite D2-630
   Quebec, QC, Canada
   G1V 2M2</text></label>
         </parameters>
         <pobox/>
         <ext/>
         <street>2875 boul. Laurier, suite D2-630</street>
         <locality>Quebec</locality>
         <region>QC</region>
         <code>G1V 2M2</code>
         <country>Canada</country>
       </adr>
       <tel>
         <parameters>
           <type>
             <text>work</text>
             <text>voice</text>
           </type>
         </parameters>
         <uri>tel:+1-418-656-9254;ext=102</uri>
       </tel>
       <tel>
         <parameters>
           <type>
             <text>work</text>
             <text>text</text>
             <text>voice</text>
             <text>cell</text>
             <text>video</text>
           </type>
         </parameters>
         <uri>tel:+1-418-262-6501</uri>
       </tel>
       <email>
         <parameters><type><text>work</text></type></parameters>
         <text>simon.perreault@viagenie.ca</text>
       </email>
       <geo>
         <parameters><type><text>work</text></type></parameters>
         <uri>geo:46.766336,-71.28955</uri>
       </geo>
       <key>
         <parameters><type><text>work</text></type></parameters>
         <uri>http://www.viagenie.ca/simon.perreault/simon.asc</uri>
       </key>
       <tz><text>America/Montreal</text></tz>
       <url>
         <parameters><type><text>home</text></type></parameters>
         <uri>http://nomis80.org</uri>
       </url>
     </vcard>
   </vcards>
\end{lstlisting}

\begin{lstlisting}
   <?xml version="1.0" encoding="UTF-8"?>
   <vcards xmlns="urn:ietf:params:xml:ns:vcard-4.0">
     <vcard>
       <fn><text>Simon Perreault</text></fn>
       <n>
         <surname>Perreault</surname>
         <given>Simon</given>
         <suffix>ing. jr</suffix>
         <suffix>M.Sc.</suffix>
       </n>
       <bday day="02" month="03" />
       <anniversary format="date-time">20090808T1430-0500</anniversary>
       <gender>M</gender>
       <lang pref="1">fr</lang>
       <lang pref="2">en</lang>
       <org type="work">Viagenie</org>
       <adr type="work">
         <label>Simon Perreault
   2875 boul. Laurier, suite D2-630
   Quebec, QC, Canada
   G1V 2M2</label>
         <street>2875 boul. Laurier, suite D2-630</street>
         <locality>Quebec</locality>
         <region>QC</region>
         <code>G1V 2M2</code>
         <country>Canada</country>
       </adr>
       <tel>
         <type>work</type>
         <type>voice</type>
         <uri>tel:+1-418-656-9254;ext=102</uri>
       </tel>
       <tel>
         <type>work</type>
         <type>text</type>
         <type>voice</type>
         <type>cell</type>
         <type>video</type>
         <uri>tel:+1-418-262-6501</uri>
       </tel>
       <email type="work">simon.perreault@viagenie.ca</email>
       <geo type="work">
         <uri>geo:46.766336,-71.28955</uri>
       </geo>
       <key type="work">
         <uri>http://www.viagenie.ca/simon.perreault/simon.asc</uri>
       </key>
       <tz>America/Montreal</tz>
       <url type="home">
         <uri>http://nomis80.org</uri>
       </url>
     </vcard>
   </vcards>
\end{lstlisting}

\subsection{HFactor}
Mike Amundsen defines a method to asses media types that he calls
``HFactor''.\footnote{\citeurl{http://amundsen.com/hypermedia/}{2011-12-21}} The
HFactor distinguishes different types of support for links and indicates which
of those are provided by a reviewed media type.

Amundsen did reviews of a couple of media types. Unfortunately these do not
include \texttt{vcard+xml} or \texttt{calendar+xml}. I'll try to identify the
HFactors of both here.

The different types of link support have two letter acronyms and fall in two
categories: Link support values, with the first letter ``L'' and Control data
support, first letter ``C''.

\begin{itemize}
\item Link Support for
  \begin{itemize}
  \item \texttt{LE} embedded links (HTTP GET)
  \item \texttt{LO} out-bound navigational links (HTTP GET)
  \item \texttt{LT} templated queries (HTTP GET)
  \item \texttt{LN} non-idempotent updates (HTTP POST)
  \item \texttt{LI} idempotent updates (HTTP PUT, DELETE) 
  \end{itemize}
\item Control Data Support to
  \begin{itemize}
  \item \texttt{CR} modify control data for read requests (e.g. \texttt{HTTP Accept-*} headers)
  \item \texttt{CU} modify control data for update requests (e. g. \texttt{Content-*} headers)
  \item \texttt{CM} indicate the interface method for requests (e.g. HTTP GET,POST,PUT,DELETE methods)
  \item \texttt{CL} add semantic meaning to link elements using link relations (e.g. HTML rel attribute)
  \end{itemize}
\end{itemize}


\section{Hypermedia in RESTful applications}

% Hat Kolab Hypermedia links in Kontakten, wie soll es sein mit xCard?

% http://restpatterns.org/Articles/The_Hypermedia_Scale

% http://linkednotbound.net/2010/12/01/web-linking/
% it is not sufficient for
% data to simply contain URIs for it to be “linked”. There must be a
% specification of the format that identifies those URIs as links, and either
% defines the link semantics or how they can be determined. The link might be
% part of a generic link construct like the Atom and HTML <link> elements,
% referencing a relation from the link relation registry that provides the link
% semantics. Alternatively, the link semantics might be defined in the data
% format, as was the case in the “next” property from our example.

% REST has four architectural constraints:
% separation of resource from representation,
% manipulation of resources by representations,
% self-descriptive messages, and
% hypermedia as the engine of application state.

% http://amundsen.com/hypermedia/hfactor/

% Hypermedia as the engine of application state
% http://www.infoq.com/articles/mark-baker-hypermedia

\begin{quotation}
  The model application is therefore an engine that moves from one state to the next by examining and choosing from among the alternative state transitions in the current set of representations.
\end{quotation}\cite[sec. 5.3, p.103]{Fielding2000}

\subsection{Hypermedia in OpenSocial}

Webfinger, e.g. get a profile picture from an email address

Danger: One can trigger na http request by sending an email.

\section{Selection of components}

Apache Shindig for Open Social, includes client tests

http://code.google.com/p/kolab-android/

https://evolvis.org/projects/kolab-ws/

http://packages.ubuntu.com/source/maverick/dovecot-metadata-plugin
https://launchpad.net/ubuntu/+source/dovecot-metadata-plugin/8-0ubuntu1

% Apache Felix, Jackrabbit, RESTeasy http://blog.tfd.co.uk/2011/11/25/minimalist/
% Scala Dispatch HTTP requests http://dispatch.databinder.net/Dispatch.html
% Scala JSON serialization https://github.com/debasishg/sjson
% ATOM http://abdera.apache.org/ http://www.ibm.com/developerworks/xml/library/x-atompp3/ http://www.ibm.com/developerworks/xml/library/x-tipatom4/index.html

% JSON: http://jackson.codehaus.org/ http://code.google.com/p/google-gson/
% http://microformats.org/wiki/org.microformats.hCard

% Universal ATOM client/server? http://code.google.com/p/dase/ (PHP/MySQL, Python client)
% https://github.com/arktekk/atom-client

% http://code.google.com/p/atombeat/ atombeat eXistDB, atompub, java, Uni Oxford, mostly written in XQuery, Spring based security
% http://atomserver.codehaus.org Adds non standard and not restful extensions (e.g. feed aggregation with special URLs) inspired by GData, expects a relational database
% more http://code.google.com/p/atomojo java atompub feed server on existDB 
% http://atomhopper.org 
% existDb has an own atompub impl http://exist-db.org/atompub.html


\subsection{REST framework}
Jersey recommended by \cite{Kaiser2011} above Restfulie and RESTeasy because of maturity and flexibility.

% http://www.torsten-horn.de/techdocs/jee-rest.htm RESTful Web Services mit JAX-RS und Jersey

Jersey has a atompub-contact client/server example app.

Why not Jersey in the end?
\begin{itemize}
\item JAX-RS assumes, that Paths are defined on the classes that represent the resources.
  \begin{itemize}
  \item This couples the ``location'' of a resource to its implementation.
  \item This leads to copied code. Given an URL pattern like
    \verb:/{AUTHORITY}/{COLLECTION}/{ENTRY}:. In this case the resource classes
    for authority, collection and entry would each need to parse the authority
    section of the path.
  \item If paths are not defined on resource classes, it is not possible to make use of JAX-RS' capabilities of declarative hyperlink building (@REF annotation).
  \end{itemize}
\item The dispatch to a request handler method has in our case three orthogonal
  parameters: HTTP verb, Media type, path. It would be preferable to handle
  these parameters independent of each other. The only way to handle at least
  the path dispatch separately is with the help of sub resources. This still
  leaves HTTP verb and Media type to be handled together.

  The sub resource mechanism additionally suffers from the shortcoming that it does not allow to specify an empty path.\footnote{\citeurl{http://java.net/jira/browse/JERSEY-536}{2012-01-21}} This makes it impossible to return a sub resource and annotate a method that should handle the case that no additional path elements remain to be matched.

\item Debugging is hard. It's not trivial to find out, why Jersey did not select a request handler or provider as the developer intended.
\item Jersey's parameter injection can not be used together with a dependency injection framework like Guice or Spring.
\end{itemize}

\subsection{VCard}

% http://sourceforge.net/projects/vcard4j is dead since 5
% years. http://sourceforge.net/projects/mime-dir-j forked and updated and is
% now also abandoned.
% http://sourceforge.net/projects/jpim/ dead since 2 years.
% active:
% http://code.google.com/p/android-vcard 
% http://sourceforge.net/projects/cardme/
% http://wiki.modularity.net.au/ical4j/index.php?title=VCard (easily extendable to XML, JSON)


ical4j 
best documented
best code
is used by 
most active
also supports icalendar
is immutable!!!

\section{Testing}
How to test the ReST/CardDAV interface?

% http://code.google.com/p/rest-client/
% http://bitworking.org/projects/apptestclient GUI based Atom Publishing Protocol Client
% 

% Jersey creates WADL documents for OPTION requests. http://wadl.java.net/ seems to provide clients

Portable Contacts test client at plaxo \url{http://www.plaxo.com/pdata/testClient}

\url{http://code.google.com/p/rest-assured/} \url{http://restfuse.com/}



\section{Standards}
\subsection{Contacts / Persons}

% http://schema.org/Person

% http://www.ibiblio.org/hhalpin/homepage/notes/vcardtable.html
\begin{description}[\breaklabel\setleftmargin{1ex}]

  \item[RFC 6450 vCard Format Specification]
    This document defines the vCard data format for representing and exchanging
    a variety of information about individuals and other entities (e.g.,
    formatted and structured name and delivery addresses, email address,
    multiple telephone numbers, photograph, logo, audio clips, etc.). This is
    the new version and obsoletes RFCs 2425, 2426, and 4770, and updates RFC
    2739.

  \item[RFC 6351 xCard: vCard XML Representation]
    This document defines the XML schema of the vCard data format. 

  % http://portablecontacts.net/draft-spec.html
  % http://docs.opensocial.org/display/OSD/Specs
  % http://docs.opensocial.org/display/OSD/Enterprise+OpenSocial+Extensions link to calendar!
  % Mozilla erwägt PoCo http://groups.google.com/group/mozilla.dev.webapi/browse_thread/thread/3bd36f73336ce783?pli=1
  % https://code.google.com/apis/contacts/docs/poco/1.0/developers_guide.html
  \item[Portable Contacts, OpenSocial] 
    Portable Contacts defines contact data structures and a ReST API. It has
    been integrated in the OpenSocial standard.

  % http://www.nuxeo.com/en/resource-center/Videos/Nuxeo-World-2011/Leveraging-Open-Social-within-the-Nuxeo-Platform
  % http://wiki.magnolia-cms.com/display/WIKI/Magnolia+OpenSocial+Container
  % http://www.zdnet.com/blog/hinchcliffe/opensocial-20-will-key-new-additions-make-it-a-prime-time-player-in-social-apps/1603
  % http://www.cmswire.com/cms/social-business/open-standards-the-future-of-opensocial-20-013335.php
  % http://docs.opensocial.org/display/OSD/List+of+OpenSocial+Containers
  % http://www.informationweek.com/thebrainyard/news/industry_analysis/232200026
  % http://www.atlassian.com/opensocial/

  \item[Nepomuk Semantic Desktop Contact Ontology]

  % http://xmlns.com/foaf/spec/
  \item[Friend of a friend (FOAF)] 
    FOAF is a 

  % http://microformats.org/wiki/hcard
  \item[hCard]

  % http://microformats.org/wiki/jcard
  \item[jCard]

\end{description}

\subsection{Calendaring}
%\subparagraph{IETF (RFC)}
\begin{description}[\breaklabel\setleftmargin{1ex}]

  \item[RFC 5545 Internet Calendaring and Scheduling Core Object Specification]

    iCalendar is the core data schema for calendaring information. This is the
    new version and obsoletes RFC2445

  \item[RFC 6321 xCal: The XML format for iCalendar]

    This specification defines a format for representing iCalendar data in
    XML. More specifically, is to define an XML format that allows iCalendar
    data to be converted to XML, and then back to iCalendar, without losing any
    semantic meaning in the data. Anyone creating XML calendar data according to
    this specification will know that their data can be converted to a valid
    iCalendar representation as well.

  \item[CalWS RESTful Web Services Protocol for Calendaring]

    This document, developed by the XML Technical Committee, specifies a RESTful
    web services Protocol for calendaring operations. This protocol has been
    contributed to OASIS WS-CALENDAR as a component of the WS-CALENDAR
    Specification under development by OASIS.

  % https://code.google.com/apis/calendar/v3
  \item[Google Calendar API V3]

    While not being a standard, the Google Calendar API is RESTful and will
    surely be implemented by many client applications. It's remarkable that the
    API supports partial GETs returning only specified fields and the HTTP PATCH
    verb to update only specified fields.

  % http://open-services.net/specifications/
  \item[Open Services for Lifecycle Collaboration (OSLC)]

    uses FOAF person \url{http://open-services.net/bin/view/Main/OSLCCoreSpecAppendixA?sortcol=table;up=#foaf_Person_Resource}

    provides change management, some overlapping to iCal TODOs \url{http://open-services.net/bin/view/Main/CmSpecificationV2}

    reference implementation: \url{http://eclipse.org/lyo}

\end{description}

\subsection{Scheduling}

\begin{description}[\breaklabel\setleftmargin{1ex}]
  \item[RFC 5546 iCalendar Transport-Independent Interoperability Protocol (iTIP)] 

    Scheduling Events, BusyTime, To-dos and Journal Entries; Specifies
    the mechanisms for calendaring event interchange between calendar
    servers. This is the new version and obsoletes RFC2446

  \item[RFC 6047 iCalendar Message-Based Interoperability Protocol (iMIP)]

    Specifies how to exchange calendaring data via e-mail. This is the new
    version and obsoletes RFC2447.

\end{description}

\subsection{Relations and Links}
% http://code.google.com/apis/socialgraph/
\begin{description}[\breaklabel\setleftmargin{1ex}]

  % http://gmpg.org/xfn/
  \item[Xhtml Friends Network (XFN)] 

    One of the relations returned by Google's webfinger.

  % https://datatracker.ietf.org/doc/draft-jones-appsawg-webfinger/
  \item[Webfinger]
    Webfinger in Firefox Contacts Add-On \url{http://mozillalabs.com/blog/2010/03/contacts-in-the-browser-0-2-released/}

  \item[RFC 6415 Web Host Metadata]

  % http://docs.oasis-open.org/xri/xrd/v1.0/xrd-1.0.html
  % http://en.wikipedia.org/wiki/XRDS
  % http://code.google.com/p/webfinger/wiki/CommonLinkRelations
  % http://hueniverse.com/category/discovery/
  \item[Extensible Resource Descriptor (XRD)] 

\end{description}

\subsection{out of scope}
\begin{description}[\breaklabel\setleftmargin{1ex}]

  % LDIF for person info

  % http://www.hr-xml.org
  % http://de.wikipedia.org/wiki/HR-XML  
  \item[HR XML]

    The HR-XML Consortium is the only independent, non-profit, volunteer-led
    organization dedicated to the development and promotion of a standard suite
    of XML specifications to enable e-business and the automation of human
    resources-related data exchanges.

  % http://www.openmobilealliance.org/Technical/release_program/cab_v1_0.aspx
  \item[OMA Converged Address Book V1.0]

    Standard by the Open Mobile Alliance defining data structures and
    synchronization of contact data. It references vCard.
  
  % http://en.wikipedia.org/wiki/Open_Collaboration_Services
  \item[Open Collaboration Services]

    Also contains data structures for persons and events but does not reuse any
    known standard. See this thread:
    \url{http://lists.freedesktop.org/archives/ocs/2011-December/000136.html}

  % http://www.w3.org/TR/contacts-api
  \item[W3C Contacts API]

    A standard on how address books cold be accessed on devices or from
    JavaScript inside a Web Browser. The standard references vCard, OMA
    Converged Address Book and Portable Contacts.

  % http://www.w3.org/TR/vcard-rdf/
  \item[W3C vCard ontology]

  % http://www.w3.org/2000/10/swap/pim/contact
  \item[W3C PIM ontology]

\end{description}


\section{People, Groups and Organizations}
% http://lists.w3.org/Archives/Public/public-device-apis/ - Contacts API
% 
% https://www.ietf.org/mailman/listinfo/calsify
% https://www.ietf.org/mailman/listinfo/ischedule - only 8 mails since 2009
% https://www.ietf.org/mailman/listinfo/httpmail only 3 mails since 2009
% https://www.ietf.org/mailman/listinfo/vcarddav
% https://www.ietf.org/mailman/listinfo/caldav
% https://www.ietf.org/mailman/listinfo/imap5

%http://groups.google.com/group/portablecontacts

%http://tech.groups.yahoo.com/group/rest-discuss

\paragraph{People}
\begin{description}[\breaklabel\setleftmargin{1ex}]

  \item[Eran Hammer-Lahav]
      \url{http://hueniverse.com}
      Yahoo!, OAuth

  \item[Eliot Lear <lear@cisco.com>]
      IETF Calsify WG chair

  \item[James Snell]
    \url{http://chmod777self.blogspot.com/}

    Apache Abdera committer, OpenSocial, IBM

  \item[Joseph Smarr]

    former Plaxo now Google
    presentation about portable contacts at vcarddav wg http://tools.ietf.org/agenda/74/slides/vcarddav-2.pdf
    http://josephsmarr.com
    http://anyasq.com/79-im-a-technical-lead-on-the-google+-team

  \item[Julian Reschke <julian.reschke@gmx.de>]
% Julian Reschke, WebDAV Experte, RFC 5995, greenbytes GmbH,Hafenweg 16, 48155 Münster , Germany

  \item[Lisa Dusseault]
      
    Lisa Dusseault is a development manager and standards architect at the Open
    Source Applications Foundation, where she's involved in the Chandler, Cosmo
    and Scooby projects. Previously, Lisa came from Xythos, an Internet startup
    where she was development manager for four years. She has also been an IETF
    contributor on various Internet applications protocols for eight years now,
    and continues to do this kind of work at OSAF. She co-chairs the IETF IMAP
    extensions and CALSIFY (Calendaring and Scheduling Standards Simplification)
    Working Groups. She is also the author of a book on WebDAV and co-author of
    CalDAV, an open and interoperable protocol for calendar access and sharing.

  \item[Mark Nottingham]
%  http://www.mnot.net/personal/

  \item[Mike Amundsen <mamund@yahoo.com>]
    \url{http://amundsen.com}

  \item[Mike Conley]

    \url{http://mikeconley.ca/blog/}
    % Email: mike.d.conley@gmail.com
    % Twitter: http://www.twitter.com/mike_conley
    % IRC: You can usually find me on Freenode as m_conley
    working on a new address book for Thunderbird: \url{https://wiki.mozilla.org/Thunderbird/tb-planning}

  \item[Peter Saint-Andre <stpeter@stpeter.im>]

    IETF Calsify WG area director

% http://notizblog.org/2011/11/17/the-long-term-failure-of-openweb/




\end{description}



\section{Implementations}

% http://wiki.portablecontacts.net/w/page/17776143/Software%20and%20Services%20using%20Portable%20Contacts
% http://docs.opensocial.org/display/OSD/List+of+OpenSocial+Containers

% http://en.wikipedia.org/wiki/List_of_applications_with_iCalendar_support
% http://syncevolution.org/
% http://www.janrain.com/solutions/supported-networks
% http://code.google.com/p/caldav4j/
% http://www.webdav.org/projects/
% http://en.wikipedia.org/wiki/CardDAV
% webdav server http://milton.ettrema.com
% http://jackrabbit.apache.org/jackrabbit-webdav-library.html
% http://davmail.sourceforge.net/ Exchange GateWay using Jackrabbit
% http://en.wikipedia.org/wiki/List_of_applications_with_iCalendar_support
% Open Core: http://en.wikipedia.org/wiki/Open_core
% http://en.wikipedia.org/wiki/Groupware
\subsection{Servers}
\begin{description}[\breaklabel\setleftmargin{1ex}]

  % http://en.wikipedia.org/wiki/Cyn.in
  \item[Cyn.in]
    Python, Open Core

  % http://www.davical.org/
  \item[DAViCal] 

    PHP, SQL storage, CalDAV, CardDav

  \item[eGroupWare]

  % http://en.wikipedia.org/wiki/EXo_Platform
  \item[eXo Platform]
    Open Core, Java, AGPL, participates in OpenSocial?

  % http://en.wikipedia.org/wiki/Group-Office
  \item[Group-Office]
    PHP, AGPL

  \item[Horde]

  % obm.org http://en.wikipedia.org/wiki/OBM_Groupware
  \item[OBM Groupware]
    PHP, GPL

  \item[Open-Xchange]
    Java, 
    In 2006 a Debian packaging attempt was canceled because upstream decided not to publish security updates for the open source version anymore.\footnote{\citeurl{http://web.archive.org/web/20100510133805/http://seraphyn.deveth.org/archives/10-Keine-Zukunft-in-der-freien-Version-von-Open-Exchange-auf-Debian.html}{2011-12-19}}

  % http://owncloud.org
  \item[owncloud]

    ownCloud supports syncing of calendar and contacts information via the
    CalDAV and CardDAV protocols.

  % http://en.wikipedia.org/wiki/Scalix
  \item[Scalix]
    Open Core
    Scalix Public License (SPL) based on MPL, requires to show the Scalix Logo

  % http://en.wikipedia.org/wiki/Simple_Groupware
  \item[Simple Groupware]
    PHP, GPL, SQL

  % http://en.wikipedia.org/wiki/SOGo
  \item[SOGo]
    CalDAV and CardDAV, written in Objective-C

  % http://en.wikipedia.org/wiki/Tiki_Wiki_CMS_Groupware
  \item[Tiki Wiki]
    PHP, SQL
    Contacts \url{http://doc.tiki.org/Contacts}, Calendar \url{http://doc.tiki.org/Calendar}
    iCal export
    apparently no CardDAV/CalDAV
    many many features!

  % http://en.wikipedia.org/wiki/Tine_2.0
  \item[Tine 2.0]
    Tine is not eGroupWare

  % http://en.wikipedia.org/wiki/Zarafa_%28software%29
  \item[Zarafa]
     PHP, MySQL
     IIRC it uses an Entity-Attribute-Value pattern to store its data in the relational db.

  % http://en.wikipedia.org/wiki/Zimbra
  \item[Zimbra]
    Open Core, Own license (Zimbra Public License),
    RFP since 2008 open: http://bugs.debian.org/cgi-bin/bugreport.cgi?bug=498316
    

\end{description}

\subsection{Clients}

\begin{description}[\breaklabel\setleftmargin{1ex}]

  % http://en.wikipedia.org/wiki/Spicebird
  \item[Spicebird]
    built on top of Thunderbird with Calendar

  \item[Thunderbird]

    CardDAV via SoCO connector \url{http://www.sogo.nu/fr/downloads/frontends.html}

  \item[WebiCal]
   % http://code.google.com/p/webical/
     Java, YUI, Web frontend for a CalDAV server, uses iCal4J

  \item[Evolution, Evolution Data Server]
  \item[KDE Kontact, Akonadi]

  \item[more CardDAV] \url{http://wiki.davical.org/w/CardDAV/Clients} \url{http://en.wikipedia.org/wiki/CardDAV#Implementations}
  \item[more CalDAV]  \url{http://wiki.davical.org/w/CalDAV_Clients} \url{http://en.wikipedia.org/wiki/CalDAV#Implementations}

\end{description}


\subsection{Web Services}
% Google Calendar http://code.google.com/apis/calendar/caldav/

\subsection{Others}

\section{Links}

\begin{itemize}
\item \url{http://thesocialweb.tv}
\item \url{http://www.vogella.de/articles/REST/article.html} REST with Java (JAX-RS) using Jersey - Tutorial
\item \url{https://addons.mozilla.org/de/firefox/addon/restclient/}
\item \url{http://dataportability.org/} still active?
\item \url{http://tech.groups.yahoo.com/group/rest-discuss/messages/17242?threaded=1&m=e&var=1&tidx=1} REST and Semantic
\item \url{http://stackoverflow.com/questions/2669926/practical-advice-on-using-jersey-and-guice-for-restful-service}
\item \url{http://macstrac.blogspot.com/2009/01/jax-rs-as-one-web-framework-to-rule.html}
\item \href{http://keithp.com/blogs/calypso/}{Calypso — CalDAV/CardDAV/WebDAV for Android and Evolution}
\item \url{http://www.xfront.com/files/articles-1.html}
\item \url{http://buzzword.org.uk/swignition/uf}
\item \url{http://json-schema.org/}
\item \href{http://www.rddl.org/}{Resource Directory Description Language (RDDL)}
\item \url{http://blogs.oracle.com/sandoz/entry/jersey_and_abdera_with_a}  \url{http://weblogs.java.net/blog/mhadley/archive/2008/02/integrating_jer_2.html}
% http://exist.sourceforge.net/
% http://wiki.davical.org/w/CardDAV/Configuration/Well-known_URLs
% https://github.com/karl/monket-google-calendar A simplified UI for Google Calendar.
% Nuxeo switches from Python to Java: http://www.infoq.com/articles/nuxeo_python_to_java http://www.infoq.com/news/nuxeo-zope-java-migration
% JAXB Tutorial http://docs.oracle.com/cd/E17802_01/webservices/webservices/docs/1.6/tutorial/doc/JAXBWorks2.html
% XML Schema http://www.javaworld.com/javaworld/jw-08-2005/jw-0808-xml.html?page=2
% https://github.com/jaliss/securesocial provides OAuth, OAuth2 and OpenID authentication for Play Framework
% Oauth http://code.google.com/intl/de/apis/accounts/docs/OAuth2.html
% Permissions compared. IMAP, WEBDAV, ... http://chandlerproject.org/bin/view/Journal/LisaDusseault20040409
% Blog on calendar interop http://calendarswamp.blogspot.com

\end{itemize}

IANA link relations registry \url{http://www.iana.org/assignments/link-relations/link-relations.xml}

\subsection{ATOM}
ATOM landscape overview \url{http://dret.typepad.com/dretblog/atom-landscape.html}
WebDAV vs. ATOM:
\url{http://intertwingly.net/wiki/pie/WebDav}
\url{http://intertwingly.net/wiki/pie/WebDavVsAtom}
google webdav atom
Why didn't ATOM succeed (more)? \citeurl{http://bitworking.org/news/425/atompub-is-a-failure}{2012-01-06}
% http://swordapp.org/

\subsection{XML and JSON}

\begin{itemize}
\item \url{http://blog.jclark.com/}
\item \url{http://code.google.com/p/jaql/wiki/Builtin_functions#xml}
\item \url{http://www.webmasterworld.com/xml/3603303.htm}
\item \url{http://www.xml.com/pub/a/2006/05/31/converting-between-xml-and-json.html?page=3}
\item \url{http://goessner.net/download/prj/jsonxml/}
\item \url{http://www.w3.org/2011/10/integration-workshop/agenda.html}
\item \url{http://jsonml.org/}
\end{itemize}

\subsection{Apache Shindig}
RPC vs. REST API for Shindig/OpenSocial: \url{http://groups.google.com/group/opensocial-and-gadgets-spec/browse_thread/thread/a4ddf7cd09f90237/5cfa1658e1c1d698?lnk=gst&q=rest#5cfa1658e1c1d698}, \url{http://groups.google.com/group/opensocial-and-gadgets-spec/browse_thread/thread/d1a5627fb6e686ce/d27d47dee92a87b2} One argument was support for batching. A restful batching proposal didn't get consensus: \url{https://docs.google.com/View?docid=dc43mmng_23fdbpp7hd&pli=1}

Flow of REST requests in Shindig \url{https://sites.google.com/site/opensocialarticles/Home/shindig-rest-java}

Google+ is likely to become OpenSocial enabled: \url{http://groups.google.com/group/opensocial-and-gadgets-spec/browse_thread/thread/1187241df6759a9a}

Shindig issues to implement OpenSocial 2.0 \url{https://docs.google.com/spreadsheet/ccc?key=0AihdZBncP3KzdGN3dVl3MFpIUlk2TXIyR3hfUDhHZUE&hl=en_US#gid=0}

How Shindig supports extensions: \url{https://cwiki.apache.org/confluence/display/SHINDIG/Arbitrary+Extensions+to+Apache+Shindig%27s+Data+Model}

Videos about some 2.0 OS features \url{http://groups.google.com/group/opensocial-and-gadgets-spec/browse_thread/thread/7b911edfb1bb3b4d}

OS and RDF \url{http://groups.google.com/group/opensocial-and-gadgets-spec/browse_thread/thread/20f62d627003509b}

OpenSocial Development Environment (OSDE, Eclipse Plugin)  \url{https://sites.google.com/site/opensocialdevenv}

\url{https://cwiki.apache.org/confluence/display/SHINDIG/Providing+your+own+data+service+implementation}

\subsection{Socialsite}

Oracle's (former Sun's) extension to Apache Shindig. Blog \url{http://blogs.oracle.com/socialsite}

\section{TODO}
\begin{itemize}
\item Does funambol.org has interesting implementations?
\end{itemize}
% LDAP wäre auch eine Möglichkeit für Addressdaten, aber:
% kein Standardschema, Mozilla anders als Apple
% manche Clients können evtl. Adressen aus LDAP lesen aber nicht schreiben.

% Calendaring is not easy as can be seen by the impressive list of failed projects:
% http://www.hula-project.org/ 
% Dreaming in Code - Scott Rosenberg's software epic. about the chandler failure
% http://xmpp.org/extensions/xep-0054.html

% http://en.wikibooks.org/wiki/LaTeX/Glossary

support Plain Text Format (text/plain), RFC5147 URI fragment identifier for plain text?

\begin{quote}
  Sowohl Atom als auch AtomPub definieren XML-Vokabulare, die eine Erweiterung
  mit zwei Mechanismen unterstützen. Zum einen ist im Standard definiert, dass
  neue Elemente in diesen Vokaularen selbst von standardkonformen Prozessoren
  ignoeriert werden müssen. [...] Gleichzeitig ist es überall dort, wo es nicht
  explizit verboten ist, möglich, Elemente ausanderen XML-Namespaces
  einzubetten.
\end{quote}\cite[p. 102]{Tilkov2011}


\bibliography{references}{}
\bibliographystyle{alphadin}
\end{document}

% Local Variables:
% ispell-dictionary: "american"
% eval: (progn (flyspell-mode 1) (outline-minor-mode 1) (goto-address-mode 1) (hide-body))
% End:
%  LocalWords:  RESTful programmatically
